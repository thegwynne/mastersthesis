 \documentclass[a4paper,12pt]{article}

\usepackage{amsmath,amssymb,graphicx,hyperref,parskip,braket}
\numberwithin{equation}{section}
\usepackage[cm]{fullpage}
\begin{document}

\title{An Introduction to the Bosonic String}

\author{Matthew Gwynne}

\maketitle
\begin{abstract}
We follow the derivation of the dynamics of the massless relativistic open string from the 1972 paper by Goddard, Goldstone, Rebbi and Thorn. Without specifying the dimension of spacetime, we begin by considering a classical string, and use a Lorentz invariant action to find a general solution to the equations of motion. In particular we show that only the transverse motion is significant, and that the ends of the string move transverse to the string at the speed of light. After fixing the light cone gauge, we quantize the string, finding that the fundamental operators are a pair of light cone position and momenta $(\hat{x}_0^-, \hat{p}^+)$, pairs of position and momenta for the transverse coordinates $(\hat{x}_0^i, \hat{p}^i)$ and an infinite set of oscillators $\hat{\alpha}_n^i$. We then show that the Virasoro algebra arises in the context of string theory through the 'minus' coordinate oscillators. We conclude by using Lorentz invariance to show that the dimension of spacetime is fixed by bosonic string theory to be 26.

\end{abstract}
\tableofcontents

\section{Introduction}
\subsection{Context and Motivation}
During the 1950s experimental physicists discovered vast numbers of new particles, called hadrons. These could be classified by a number of properties (charge, spin, ’strangeness’), and led to the development of what we now call the Standard Model. The Standard Model can be thought of as the combination of quantum chromodynamics (the types of particle) and the electroweak interaction, which is the best known description of quantized electromagnetism and the weak force, two of the four fundamental forces. The Standard Model remains to this day the best picture we have of the fundamental physics of the universe, but it is also incomplete. For one thing, it has a large number of dimensionless parameters that can’t be calculated, and must be fixed manually. This is an issue as it makes the theory rather arbitrary; with a different choice for these parameters one would obtain a different form of the standard model. Secondly, it doesn’t contain gravity. Attempts have been made to incorporate an additional particle, dubbed a ’graviton’ into the standard model, but even these attempts have so far failed to incorporate general relativity. Partially this is due to the fundamentally classical nature of general relativity, as opposed to the fundamentally quantum nature of the Standard Model.

The first of these two concerns doesn’t necessarily have to be an issue. We are accustomed to the natural sciences having elegant corresponding mathematical analogues, but the possibility of certain physical properties being arbitrary is worth considering. On the other hand, no serious physical theory can hope to get away with not including gravity. While on the small scale where gravity is negligible the standard model works effectively, on cosmological scales it simply fails to give useful predictions due to this omission. At this stage physicists set out to try to find some kind of unification for all the known forms of matter and the fundamental forces. And this is the context in which string theory arises.

It is important to clarify first what string theory is not. It is not physics. One facetious argument for this can be seen by comparing the departmental research web pages of various universities’ ’Mathematical Physics’ groups and ’Physics’ groups. Many mathematical physics groups will mention string theory as a research area, while almost no physics departments do. This actually conveys a genuine feature of string theory; it has yet to make any real falsifiable predictions, and as such isn’t
really part of science. This means it is of limited interest to a physicist. To a mathematician however it is fascinating. It becomes almost irrelevant if the theory is ’correct’ or not, or even if it has any non-trivial correlation to the physical world. To a mathematician it is interesting enough that it contains a number of interesting mathematical features. The generalised covariant action of the string is a problem of variational calculus, the quantization of the string turns out to be an example of a Lie algebra, and several other seemingly disconnected areas of mathematics arise in the string’s formulation.

Now, what string theory is. The basic concept is that instead of the universe consisting of point-like particles, matter consists of tiny pieces of string moving in spacetime. The use of the word string allows this to make sense to the average non-mathematician, but in essence all we are doing is replacing zero dimensional objects with one dimensional objects, with all that that entails. In fact, what this entails looks quite like physics as we understand it. From this starting point it is possible to develop all of the particles that we consider to be part of the standard model, and other features (such as electromagnetism) that we would hope to find in a valid physical theory. It is however necessary to make some unusual intuitive leaps, most notably that it requires additional spatial dimensions than the 3 that we have observed.

But the fact that it potentially represents the standard model isn’t enough to make string theory interesting. What really makes it stand out as a physical theory is that it smoothly reconciles general relativity with quantum mechanics, and it does it by accident. It was never constructed in order to make this happen, but after the construction, at low energies quantum gravity appeared as an emergent phenomenon.

The characteristics of a point particle moving in space can naturally be considered by plotting its position in space at each moment in time in a 4-dimensional vector space. We call this 1-dimensional trajectory the ’worldline’ of the particle in spacetime. This is then a bijective relationship. The worldline contains complete information about the history of the particle. Analogously, the motion of a 1-dimensional string moving in spacetime corresponds to a ’worldsheet’, a 2-dimensional surface which also contains complete information about the motion of the string. The different vibrations (modes) of the string appear in the worldsheet as waves on the surface, and correspond to different particles.
Strings can be either open or closed, depending on whether the end points of the string coincide. It is possible to construct string theories that consist of both open and closed strings, or that only contain closed strings, but as any open string can become closed by connecting its end points, it is impossible to construct a string theory containing only open strings. For an open string the worldsheet will look like a sheet (as might be expected). For a closed string it will look more like a cylinder.
\subsection{Prerequisite Knowledge and Core Texts}
Due to the nature of this topic a fairly advanced knowledge of quantum mechanics and relativity theory is needed in order to fully understand the concepts in this project. Many techniques in the development of string theory have analogues in analytical mechanics and quantum field theory. As often as possible, if a result from outside of the project is used it will come along with a reference for a textbook appropriate for further reading. In most of these cases the results are fairly central to the subject, so other textbooks on the same area will likely also include the same content.

For a less thorough understanding further reading should be unnecessary. It is fully possible to follow much of narrative of the project by simply taking the quoted results on trust. 

The core structure of the project will follow the route laid down in the seminal paper \emph{Quantum Dynamics of a Massless Relativistic String} by Goddard, Goldstone, Rebbi and Thorn (GGRT)\cite{ggrt}. This was the original paper in which the authors derived the equations of motion and quantization of the bosonic string. That paper concluded by demonstrating that the dimension of spacetime required by bosonic string theory is 26 (one time and 25 spacial dimensions), and this is also the main result of this project.

As the study of string theory is now over 40 years old, there is a sizeable body of established literature in the area\cite{polchinski}\cite{gsw}, as well as courses taught at some universities with lecture notes available online\cite{tong}\cite{thooft}. The core reference textbook used for this project is \emph{A First Course in String Theory} by Barton Zwiebach\cite{zwiebach}, and some of the more tricky technical derivations in this project (such as the commutation relations from section \ref{secvirasoro}) follow that text.
\subsection{Project Overview} 
This project is split into four major sections, followed by a sort of epilogue containing some interesting additional results.

The first section contains useful preliminary knowledge. Following the structure of GGRT, we begin by discussing the non-relativistic string (section \ref{nonrelativistic}). We then find a relativistic action for first the point particle (\ref{sectionpoint}), followed by the string (\ref{sectionstringaction}). In the second section we use this action to learn about the classical dynamics of the string (\ref{sectioneom}), eventually arriving at equations for the motion in a particularly useful choice of gauge (\ref{lceom}).

Before attempting to quantize the string we first investigate the quantization of the point particle (\ref{secpoint}). We then use the insights gained to quantize the string (\ref{qrs}), and in the process we arrive at the major result of the project, namely the dimensionality of spacetime (\ref{dimension}).
\section{Preliminaries}\label{prelim}
\subsection{The Classical Non Relativistic String}\label{nonrelativistic}
To begin with we reduce the problem as much as possible, and determine the dynamics of a classical, non-relativistic string. To do this, we need to construct the Lagrangian, and hence the kinetic and potential energies. Consider a string with mass per unit length $\rho$ and tension $T_0$ in two-dimensions $(x,y)$, stretched between the points $(0,0)$ and $(0,l)$. With the two endpoints fixed, we can say that the x direction is longitudinal, and the y direction is transverse. The kinetic energy is simple, for each infinitesimal segment $dx$ (which has mass $\rho dx$), its velocity is $\partial y/ \partial t$, so summing over all such segments gives 
\begin{equation}\label{nrkinetic}
T =  \int_0^l \frac{1}{2}  \rho  \left( \frac{\partial y}{\partial t} \right)^2 dx.
\end{equation}
The potential energy comes from the work done to stretch each infinitesimal segment. Consider one segment which, when stationary, stretches between $(x,0)$ and $(x + dx, 0)$. If the right-hand endpoint of this string is moved from $(x+dx, 0)$ to $(x+dx, dy)$ then the change in length of the segment, $\Delta$, is given by 
\begin{equation}
\Delta = \sqrt{(dx)^2 + (dy)^2} - dx.
\end{equation} 
We now assume that the oscillations are small, and so can use the first two terms of the Taylor expansion of the square root, so
\begin{equation}
\Delta \approx \frac{1}{2}dx \left( \frac{\partial y}{\partial x} \right)^2.
\end{equation}
Now, by Hooke's law, the work done for each segment is $T_0 \Delta$, so the total potential energy of the string is 
\begin{equation}
V = \int_0^l \frac{1}{2} T_0 \left( \frac{\partial y}{\partial x} \right) ^2 dx.
\end{equation} 
Thus, the Lagrangian, $T - V$ is given by 
\begin{equation}
L = \int_0^l \frac{1}{2}\left( \rho  \left( \frac{\partial y}{\partial t} \right)^2 -  T_0 \left( \frac{\partial y}{\partial x} \right) ^2 \right) dx.
\end{equation}
For such a string moving from time $t_i$ to time $t_f$, the action is therefore given by 
\begin{equation}
S = \int_{t_i}^{t_f} \left[ \int_0^a \frac{1}{2}  \left( \rho  \left( \frac{\partial y}{\partial t} \right)^2 -   T_0 \left( \frac{\partial y}{\partial x} \right) ^2 \right) dx \right] dt.
\end{equation}
Varying this action gives 
\begin{equation}
\delta S = \int_0^l \left[ \rho \frac{\partial y}{\partial t} \delta y\right]_{t_i}^{t_f} dx -  \int_{t_i}^{t_f} \left[T_0\frac{\partial y}{\partial x} \delta y \right]_0^l dt - \int_{t_i}^{t_f} \left( \int_0^l \left( \rho \frac{\partial ^2 y}{\partial t^2} -T_0 \frac{\partial ^2 y}{\partial x^2}\right)\delta y dx \right) dt
\end{equation}
In order to satisfy $\delta S = 0$, all of the terms must vanish separately. The first of these is effectively satisfied by specifying the initial and final configurations of the string, so is of little interest. 
The third term gives us our string equations of motion, 
\begin{equation}\label{eom2d}
\rho \frac{\partial ^2y}{\partial t^2} = T_0 \frac{\partial ^2 y}{\partial x^2}.
\end{equation}
The second term can be expanded out to 
\begin{equation}
\int_{t_i}^{t_f}\left(T_o \frac{\partial y}{\partial x}(t,l)\delta y(t,l) - T_0 \frac{\partial y}{\partial x}(t,0)\delta y(t,0)\right) dt = 0,
\end{equation}
which makes it clear that it concerns the spatial boundary conditions (the ends of the string). Clearly, the integrand vanishes if $\partial y/\partial x =0$ at both end points. Slightly less clearly, the equality also holds if $\partial x/\partial t =0$ at both end points. These correspond to Neumann or Dirichlet boundary conditions respectively, and to decide which one is appropriate, we examine the total momentum. 
\begin{equation}
p = \int_0^l \rho \frac{\partial y}{\partial t} dx.
\end{equation}
Conservation of this momentum would be given by
\begin{equation}
\frac{dp}{dt} = 0 = \int_0^l \rho \frac{\partial ^2y}{\partial t^2} dx = \int_0^l T_0 \frac{\partial^2 y}{\partial x^2} dx,
\end{equation}
where the final equality is due to equation \ref{eom2d}. The integral sign then cancels out one of the derivatives, and the tension is strictly positive, so we have  
\begin{equation}
\left[ \frac{\partial y}{\partial x} \right]_0^l = 0.
\end{equation}
Note that this means momentum is conserved with \emph{Neumann} boundary conditions. This does not necessarily mean that Dirichlet boundary conditions are never valid, just that they don't necessarily conserve momentum within the strings. If the strings were attached to some kind of structure that had its own momentum, then overall momentum could still be conserved with Dirichlet boundary conditions\footnote{Structures of this type are called D-branes, for a detailed description see Zwiebach\cite{zwiebach}}.  For the purposes of this project, we will always assume Neumann boundary conditions.

Now this gives string equations of motion and boundary conditions, but the action we used was non-relativistic, and we must now seek a Lorentz covariant equivalent. First, however, a brief introduction to Lorentz covariance.
\subsection{Lorentz Covariance}
The core principle that led to the discovery of special relativity was the idea that the speed of light is constant, and will take the same value regardless of the motion of the observer. By making this assumption much of our intuition about the dynamics and shape of spacetime breaks down somewhat. For one thing, the distance between two objects becomes ill-defined; to an inertial observer, a ruler would appear shorter than its rest length if it was travelling sufficiently quickly.

The 'replacement' for three-dimensional distance is four-dimensional spacetime distance. Instead of considering a point in space defined by its 3 spacial coordinates, we consider an \emph{event} in spacetime defined by 3 spacial coordinates ($x^1, x^2, x^3$) , and one time coordinate ($x^0$). Then, according to special relativity, the spacetime interval between two events, defined to be 
\begin{equation}
-\Delta s^2 = -(\Delta x^0)^2 + (\Delta x^1)^2+(\Delta x^2)^2+(\Delta x^3)^2
\end{equation}
is the same for all inertial observers\footnote{The choice of having the spatial coordinates being positive is the standard convention for string theory, as in any reference frame the configuration of a string is a space-like curve}. If $\Delta s^2$ is positive, the two events are said to be timelike separated. For example if you consider the trajectory of a particle moving in space, all points along that curve are timelike separated, as massive particles can't move faster than light. This in fact is the simplest way to see that the Lagrangian in the previous section is not relativistic, as it did not have any means to rule out the possibility of faster than light travel.

Notice that in the definition of the spacetime interval the time coordinate has the opposite sign to the space coordinates. This signature is called the Minkowski metric.

The standard convention for writing down expressions involving spacetime coordinates allows for a great deal of simplification of notation. The coordinates of the event $x$ in space time are denoted by $x^\mu$ \footnote{In general Greek characters are used to denote spacetime indices, while Latin characters are used when only the spacial coordinates are included}. In this convention, the Minkowski metric can be expressed as a matrix, 
\begin{equation}
\eta _{\mu \nu} =  \begin{pmatrix} -1 & 0 & 0 & 0 \\ 0 & 1 & 0 & 0 \\ 0 & 0 & 1 & 0 \\ 0 & 0 & 0 & 1 \end{pmatrix} 
\end{equation}.

In our new convention, an expression containing the same index twice (once as a subscript and once as a superscript) in one term implicitly sums over all possible values of that index. So an expression like $x^\mu y_\mu$ is taken to mean $x^0y_0+x^1y_1 + x^2y_2 + x^3y_3$, and would resolve to a scalar quantity. Combining this idea with the Minkowski metric allows the introduction of a vector with lowered indices. The term $x_\mu$, which is defined to be $\eta_{\mu \nu} x^\nu$, has the same values as $x^\mu$, but with opposite sign on the time coordinate. This allows the space time interval to be expressed much more compactly, as using this convention we can write
\begin{equation}\label{sti}
-\Delta s ^2  = \eta_{\mu \nu} x^\nu x^\mu.
\end{equation}
Now, the feature that is significant here is that the expression has no free indices; all indices are summed over. This means that it is unchanged by Lorentz transformations and is said to be \emph{Lorentz covariant}. This is the property that we will seek for our relativistic action. An action composed of Lorentz vectors with no free indices will automatically be Lorentz covariant.

With all of this notation set up, it becomes trivial to extend space time to have any number of dimensions. Instead of having one time and three spacial coordinates, we instead have one time coordinate and $d-1$ spacial coordinates. The metric continues the same pattern, 
\begin{equation}
\eta _{\mu \nu} =  \begin{pmatrix} -1 & 0 & 0  & \dots & 0\\ 0 & 1 &  0 & \dots &0\\ 0 & 0 & 1 &\dots&0 \\  \vdots &\vdots & \vdots & \ddots&\vdots\\0 & 0 & 0 & \dots & 1 \end{pmatrix}, 
\end{equation}
and by using this matrix for $\eta_{\mu\nu}$ the expression for the spacetime interval \ref{sti} remains the same.


Before considering a relativistic string, it is instructive to see how to find a relativistic action for a point particle.

\subsection{Relativistic Point Particle}\label{sectionpoint}
\subsubsection{Relativistic Action}
As a particle moves in space, it traces out a curve in spacetime, which we shall call $C$. The action should be a functional of properties of this curve. In order to find an appropriate action, we need properties of the particle that all observers will agree on. The most natural property would be the proper time elapsed, i.e the time difference in the rest frame of the particle. Just for now, we will temporarily work in 4-dimensional space time. Infinitesimally , 
\begin{equation}
-ds^2 = c^2 dt^2 - (dx^1)^2 - (dx^2)^2 - (dx^3)^2, 
\end{equation}
and the proper time $\tau$ is equal to $ds/c$. 
This doesn't however have the correct dimensions. The action is a Lagrangian (units of energy, or $M L^2/T^2$) integrated over time, so has overall units of $M L^2/T$. If we integrate the proper time over the world line, the units will just be $T$, so we need a correcting mass and velocity squared. For the mass we can use the rest mass of the particle (call it $m$), and the only appropriate velocity is the speed of light $c$. The velocity of the particle would not be a valid choice, as it is reference frame dependent, so would not give a covariant action. 

This action turns out to be correct (apart from a minus sign), so we have
\begin{equation}\label{pointaction}
S = -\int_C mc\,\, ds 
\end{equation}
To put this action in some kind of context it is helpful to consider it in some particular Lorentz frame, relative to which the particle is moving with velocity $v$. In that frame, the proper time is related to the reference time by 
\begin{equation}
ds = c \, \, dt\sqrt{1-\frac{v^2}{c^2}},
\end{equation}
so the action becomes
\begin{equation}
S = -mc^2 \int_{t_i}^{t_f} \sqrt{1-\frac{v^2}{c^2}} dt,
\end{equation}
where $t_i$ and $t_f$ are the initial and final times in the observers frame of reference. This gives us a Lagrangian of
\begin{equation}\label{pointlagrangian}
L = -mc^2\sqrt{1-\frac{v^2}{c^2}}.
\end{equation}

Notice that with low speeds, using the binomial expansion of the square root and discarding any times higher than $2^{nd}$ order in the velocity, we recover the non relativistic Lagrangian, 
\begin{equation}
L \approx - mc^2 + \frac{1}{2} m v^2.
\end{equation} 
This should be encouraging that the action is doing what it is supposed to do. 
\subsubsection{Equations of Motion}
We now proceed by standard analytical mechanics techniques\cite{analytical}, and vary the action to find the equations of motion. In order to find the correct variation of $ds$, it is much simpler to work with the variation of $ds^2$.

Suppose the curve $C(\tau)$ parameterizes the worldline of the particle by the proper time, which we may freely do due to reparameterization invariance. We can write
\begin{equation}
ds^2 = -\eta_{\mu \nu} dx^\mu dx^\nu = -\eta_{\mu \nu} \frac{dx^\mu}{d\tau}\frac{dx^\nu}{d\tau}(d\tau)^2.
\end{equation}
Varying both sides of this expression gives
\begin{equation}
2ds\delta(ds)=2\eta_{\mu \nu} \delta\left(\frac{dx^\mu}{d\tau}\right)\frac{dx^\nu}{d\tau}(d\tau)^2,
\end{equation}
where both sides are calculated using a simple application of the chain rule. Rearranging to make $\delta(ds)$ the subject and simplifying gives
\begin{equation}
\delta(ds) = \eta_{\mu \nu} \frac{d(\delta x^\mu)}{d\tau}\frac{dx^\nu}{ds}\,d\tau.
\end{equation}
Using this and equation \ref{pointaction} we have
\begin{equation}
\delta S = -mc\int_C \frac{d(\delta x^\mu)}{d\tau}\frac{dx_\mu}{ds}\,d\tau,
\end{equation}
where the $\eta_{\mu \nu}$ has been used to lower the index on the second component of the integrand.
Currently, there are still derivatives acting on the variation of the coordinate, so we use the product rule to separate it into a $\tau$ derivative of the entire integrand, and a term where $\delta(x^\mu)$ appears without any derivatives.
\begin{equation}\label{pointseparated}
\delta S = -mc\int_C \frac{d}{d\tau}\left(\delta x^\mu \frac{dx_\mu}{ds}\right)\,d\tau + mc\int_C\delta x^\mu \frac{d}{d\tau}\left(\frac{dx_\mu}{ds}\right) \, d\tau
\end{equation}
The first term collapses by the fundamental theorem of calculus into the values of $\delta x^\mu (dx_\mu/ds)$ evaluated at the boundaries of $C$ (the initial and final values of $\tau$), and as such vanishes as these values can be fixed. Then, by setting $\delta S = 0$ the equations of motion are determined by whatever is left in the second term apart from the $\delta x^\mu$.
\begin{equation}
mc\frac{d}{d\tau}\left(\frac{dx_\mu}{ds}\right) = \frac{d}{d\tau}\left(mc\frac{dx_\mu}{d\tau}\right) = \frac{dp_\mu}{d\tau} = 0
\end{equation}
Thus our equations of motion are that each component of the momentum of the particle is conserved by its motion, and this statement is not frame dependent.

In this case the two Lorentz-vectors that were contracted together happened to be the same vector. This 'squaring' isn't the only way to construct a Lorentz covariant action. For example, if we were modelling a charged particle interacting with an electric field the interaction term in the action would be linear in the particle velocity, which is contracted against the field potential $A_\mu$. In this project all action terms that we consider will be squared vectors, but this is not always true in general. 
\subsection{Relativistic Strings}\label{sectionstringaction}
When we were considering a relativistic particle, the goal was to try to find a Lorentz covariant action based on the particle's world line. Now that we are working with strings, the subset of spacetime swept out by the string is now a 2-dimensional sheet, but the goal of finding a Lorentz covariant action remains the same. 

In the case of the point particle, the action turned out to be constructed most simply out of the proper time along the worldline, so the natural place to look is in some kind of 'proper area'. Differential geometry gives us the tools required to find the area of a surface. For ease of understanding we will first consider spatial surfaces in three dimensions.
\subsubsection{Spacial Surfaces}
A surface can (at least locally) be parameterized by two parameters, which we will call $\xi_1$ and  $\xi_2$. The set of parameters take values on the \emph{coordinate space}, and a set of functions which we call $X^i$ provide an isomorphism between the coordinate space and a section of the surface. We will call the space in which the surface lives the \emph{target space}. The $X^i$ are thus functions of $\xi_1$ and $\xi_2$.

In general, an infinitesimal rectangular interval in the coordinate space will correspond to an infinitesimal trapezium in the target space. The area $A$ of a parallelogram spanned by two edges of length $a, b$ at an angle $\theta$ to each other is given by $A = a \,b \sin \theta$. If we call the vectors that span the area element in the target space $da$ and $db$, then
\begin{equation}\label{edges}
da^i = \frac{dX^i}{d\xi^1}d\xi^1,\quad db^i= \frac{dX^i}{d\xi^2}d\xi^2.
\end{equation}
Using these, we can construct the area element $dA$ in the target space.
\begin{eqnarray*}
dA & = & |da||db|\sin\theta \\
& = & |da||db|\sqrt{1-\cos^2\theta} \\
& = & |da||db|\sqrt{1 - \left(\frac{da.db}{|da||db|}\right)^2} \\
& = & \sqrt{|da|^2|db|^2 - (da.db)^2} \\
& = & \sqrt{(da.da)(db.db) - (da.db)^2}
\end{eqnarray*}
In the third line of this derivation we have used the following definition of the dot product of two vectors.
\begin{equation}
\mathbf{a}.\mathbf{b} = |\mathbf{a}||\mathbf{b}|\cos\theta.
\end{equation}
Using definitions \ref{edges}, the area element can be expressed in terms of the parameters by 
\begin{equation}\label{areaelement}
dA = d\xi^1d\xi^2\sqrt{\left(\frac{dX^i}{d\xi^1}\frac{dX_i}{d\xi^1}\right)\left(\frac{dX^i}{d\xi^2}\frac{dX_i}{d\xi^2}\right) - \left(\frac{dX^i}{d\xi^1}\frac{dX_i}{d\xi^2}\right)^2},
\end{equation} 
and the total area is given by the integral of this quantity over the coordinate space.

We want to demonstrate that this quantity is reparameterization invariant. For the point particle this was trivial; as there is only one coordinate along the string (say $\tau$), any reparameterization would be a function of the one coordinate, $\tilde{\tau}(\tau)$. This meant that the Jacobian and inverse Jacobian would immediately cancel.

In this case however, we must assume that the two coordinates would mix up, $\tilde{\xi}^1=\tilde{\xi}^1(\xi^1, \xi^2)$, $\tilde{\xi}^2 = \tilde{\xi}^2(\xi^1,\xi^2)$, so it is non-trivial to demonstrate the reparameterization invariance. The simplest way is to construct the induced metric $g_{i j}$, defined by
\begin{equation}
g_{i j } = \frac{dX^k}{d\xi^i}\frac{dX_k}{d\xi^j}.
\end{equation}
Notice that, considering $g_{ij}$ as a matrix, $\det(g_{i j})$ is given by the expression under the square root in equation \ref{areaelement}. If we say $g = \det(g_{i j})$, then we can write
\begin{equation}\label{gaction}
A = \int d\xi^1d\xi^2\sqrt{g}.
\end{equation}
Now, to demonstrate reparameterization invariance, we construct the spacial interval in terms of this new construction. Now, remembering that we are still working with all spatial coordinates, $ds^2 = dX^kdX_k$, and we can express $dX^k$ in terms of the differentials $d\xi^1, d\xi^2$ by
\begin{equation}
dX^k = \frac{dX^k}{d\xi^1}d\xi^1 + \frac{dX^k}{d\xi^2}d\xi^2 = \frac{dX^k}{d\xi^i}d\xi^i,
\end{equation}
where the index i takes the values $1, 2$.
Using this, we can see 
\begin{equation}
ds^2 = \frac{dX^k}{d\xi^i}\frac{dX_k}{d\xi^j}d\xi^id\xi^j = g_{i j}d\xi^id\xi^j.
\end{equation}
Thus, if we have two different parameterizations $(\xi^1, \xi^2)$ and $(\tilde{\xi^1}, \tilde{\xi^2})$, with induced metrics $g_{i j}$ and $\tilde{g}_{i j}$ respectively, the invariance of the spacetime interval gives
\begin{equation}
g_{i j }\, d\xi^i d\xi^j = \tilde{g}_{k l}\, d\tilde{\xi}^k d\tilde{\xi}^l.
\end{equation}
We want to rearrange this into an expression for one of the metrics in terms of the differentials from each parameterization. We can do this using the chain rule, as
\begin{equation}
d\tilde{\xi}^i = \frac{\partial \tilde{\xi}^i}{\partial \xi^j}d\xi^j,
\end{equation}
and then the differentials cancel, leaving 
\begin{equation}\label{gij}
g_{i j} = \tilde{g}_{k l} \frac{\partial \tilde{\xi}^k}{\partial \xi^i}\frac{\partial \tilde{\xi}^l}{\partial \xi^j}.
\end{equation}
Considering the combinations $\partial{\xi}^i/\partial{\xi}^j$ as matrices, we can rearrange to get
\begin{equation}\label{gijdet}
g_{i j} = \left(\frac{\partial \xi^i}{\partial \tilde{\xi}^k}\right)^T\tilde{g}_{k l} \frac{\partial \tilde{\xi}^l}{\partial \xi^j}.
\end{equation}
Where $M^T$ denotes the matrix transpose of $M$.

Now in multivariable calculus it is a standard result that a change of variables 
\begin{eqnarray*}
\xi^1 \to \tilde{\xi}^1\\
\xi^2 \to \tilde{\xi}^2
\end{eqnarray*}
results in a change of differentials 
\begin{equation}
d\xi^1d\xi^2 = \left|\det\left(\frac{\partial\xi^i}{\partial\tilde{\xi}^j}\right)\right|d\tilde{\xi}^1d\tilde{\xi}^2
\end{equation}
and that the inverse change of variables gives
\begin{equation}
d\tilde{\xi}^1d\tilde{\xi}^2 = \left|\det\left(\frac{\partial\tilde{\xi}^k}{\partial\xi^l}\right)\right|d\xi^1d\xi^2.
\end{equation}
Together these equations say that
\begin{equation}
\Big|\det\Big(\frac{\partial\xi^i}{\partial\tilde{\xi}^j}\Big)\Big| \Big|\det\Big(\frac{\partial\tilde{\xi}^k}{\partial\xi^l}\Big)\Big|=1.
\end{equation}
Now if we take determinants of equation \ref{gijdet} we get
\begin{eqnarray}
g &=& \Big|\det\Big(\left(\frac{\partial\xi^i}{\partial\tilde{\xi}^k}\right)^T\Big)\Big| \tilde{g}\Big|\det\Big(\frac{\partial\tilde{\xi}^l}{\partial\xi^j}\Big)\Big|\\
&=& \Big|\det\Big(\frac{\partial\xi^i}{\partial\tilde{\xi}^k}\Big)\Big| \tilde{g}\Big|\det\Big(\frac{\partial\tilde{\xi}^l}{\partial\xi^j}\Big)\Big|\\
&=& \tilde{g}
\end{eqnarray}
So $g$ is invariant under a reparameterization, and our action is reparameterization invariant.
\subsubsection{Spacetime Surfaces}
This has given us a reparameterization invariant functional for the area of a surface in space, but for strings the worldsheet is a surface in spacetime. In particular, at all points on the worldsheet, there must be at least one tangent vector pointing in a timelike direction (the worldline of 'one point\footnote{This is technically not a valid statement, as due to the lack of substructure of the string it is impossible to track the motion of any point except an end point. For more detail see section 7.1 of Zwiebach. }' on the string in an inertial frame), and at least one tangent vector pointing in a spacelike direction (the configuration of the string at a fixed point in time in an inertial frame). 

By convention, we will use the names $\sigma$ and $\tau$ instead of $\xi^1$ and $\xi^2$. Thus, the worldsheet is now given by the set of functions $X^\mu(\tau, \sigma)$. The natural extension of the area functional is formed by simply adding the $X^0$ coordinate back into equation \ref{edges}.
\begin{equation}
da^\mu = \frac{dX^\mu}{d\tau}d\tau, \quad db^\mu = \frac{dX^\mu}{d\sigma}d\sigma.
\end{equation}
Using this to construct the area functional however takes a little bit more thought than in the spacial case. This is because the expression under the square root in equation \ref{areaelement} is not obviously positive in the spacetime case. In order to ensure it has a fixed sign, we use the fact that the tangent space of every point on the worldsheet is spanned by a spacelike and a timelike vector. To see this, consider the set of tangent vectors at a point given by parameters $\tau = \tau_0$, $\sigma = \sigma_0$ (although the values of the coordinates in the equations are omitted for readability), 
\begin{equation}
v^\mu(t) = \frac{\partial X ^ \mu}{\partial \tau} + t\frac{\partial X^\mu}{\partial \sigma},
\end{equation}
for $t \in (-\infty, \infty)$. This gives something approximating polar coordinates; as $t$ varies you get a sort of circle of vectors. Squaring vectors in this set gives the quadratic equation
\begin{equation}
v^\mu(t) v_\mu  (t) = t^2\left(\frac{\partial X^\mu}{\partial \sigma}\frac{\partial X_\mu}{\partial \sigma}\right) + 2t\left(\frac{\partial X^\mu}{\partial\tau}\frac{\partial X_\mu}{\partial\sigma}\right) + \left(\frac{\partial X^\mu}{\partial \tau}\frac{\partial X_\mu}{\partial \tau}\right).
\end{equation}
If we require the tangent space to require both timelike and spacelike vectors, that means that this expression must be capable of people either positive or negative. This means the equation
\begin{equation}
v^\mu(t)v_\mu(t) = 0
\end{equation}
must have two real roots, so the discriminant, 
\begin{equation}
\left(\frac{\partial X^\mu}{\partial \tau}\frac{\partial X_\mu}{\partial \sigma}\right)^2 - \left(\frac{\partial X^\mu}{\partial \sigma}\frac{\partial X_\mu}{\partial \sigma}\right)\left(\frac{\partial X^\nu}{\partial \tau}\frac{\partial X_\nu}{\partial \tau}\right) > 0.
\end{equation}

Since this quantity is positive, we see that in order to construct an area functional, we must introduce a minus sign under the square root in equation \ref{areaelement}. 

Our relativistic string action is given by
\begin{equation}
S = - \frac{-T_0}{c} \int_{\tau_i}^{\tau_f}d\tau \int_{\sigma_i}^{\sigma_f} d\sigma \sqrt{\left(\frac{\partial X^\mu}{\partial \tau}\frac{\partial X_\mu}{\partial \sigma}\right)^2 - \left(\frac{\partial X^\mu}{\partial \sigma}\frac{\partial X_\mu}{\partial \sigma}\right)\left(\frac{\partial X^\nu}{\partial \tau}\frac{\partial X_\nu}{\partial \tau}\right)}.
\end{equation}
This action is constructed from Lorentz vectors, and inherits reparameterization invariance from the spatial version. To find the constants, we treated $\tau$ as having the dimension of $T$, and $\sigma$ as having the dimension of $L$. Then, since the area has dimension of $L^2$, we multiplied by tension/velocity to achieve the desired units of $ML^2/T$. 

This expression can be simplified, making use of 
\begin{equation}
\dot{X}^\mu =\frac{\partial X^\mu}{\partial \tau}, \quad X'^\mu = \frac{\partial X^\mu}{\partial \sigma},  
\end{equation}
to allow us to write
\begin{equation}\label{contractedaction}
S = \frac{- T_0}{c} \int_{\tau_i}^{\tau_f} d\tau \int_{\sigma_i}^{\sigma_f} d\sigma \sqrt{(\dot{X}.X')^2 - (\dot{X})^2(X')^2}.
\end{equation}
In order to get this action to look like equation \ref{gaction}, we define the induced metric on the target Minkowski space, 
\begin{equation}
\gamma_{\mu \nu} = \begin{pmatrix} 
(\dot{X})^2 & \dot{X}.X' \\
 \dot{X}.X' & (X')^2 \end{pmatrix}.
\end{equation} 
If we similarly use $\gamma = \det(\gamma_{\mu \nu})$, we can rewrite the action as 
\begin{equation}
S = \frac{-T_0}{c}\int d\tau d\sigma \sqrt{-\gamma},
\end{equation}
where the integration is done over the boundary. Conventionally, the $\sigma$ parameterization is chosen to be between $0$ and $\pi$ for open strings. This action is called the \emph{Nambu-Goto action}.
\section{Equations of Motion}\label{sectioneom}
Deriving the string equations of motion from \ref{contractedaction} is relatively straightforward and, similarly to the point particle, follows the standard analytical mechanics procedure. The Lagrangian density is given by 
\begin{equation}
\mathcal{L}(\dot{X}, X') = \frac{-T_0}{c} \sqrt{(\dot{X}.X')^2 - (\dot{X})^2(X')^2},
\end{equation}
which allows us to write the variation 
\begin{eqnarray*}
\delta S & = & \int_{\tau_i}^{\tau_f} d\tau \int_0^\pi d\sigma \delta\left(\mathcal{L}(\dot{X}, X')\right) \\
& = & \int_{\tau_i}^{\tau_f} d\tau \int_0^\pi d\sigma\left( \frac{\partial \mathcal{L}}{\partial \dot{X}^\mu}\frac{\partial (\delta X^\mu)}{\delta \tau} + \frac{\partial \mathcal{L}}{\partial \dot{X}^\mu}\frac{\partial (\delta X^\mu)}{\delta \sigma} \right)
\end{eqnarray*}
The first half of each of these two terms can be calculated explicitly by a simple application of the chain rule, giving
\begin{eqnarray}
\frac{\partial \mathcal{L}}{\partial \dot{X}^\mu} & = & \frac{-T_0}{c}\left( \frac{(\dot{X}.X')X'_\mu - (X')^2\dot{X}_\mu}{\sqrt{(\dot{X}.X')^2 - (\dot{X})^2(X')^2}}\right)\label{taumomentum}, \\
\frac{\partial \mathcal{L}}{\partial X'^\mu} & = & \frac{-T_0}{c}\left( \frac{(\dot{X}.X')\dot{X}_\mu - (X')^2X'_\mu}{\sqrt{(\dot{X}.X')^2 - (\dot{X})^2(X')^2}}\right)\label{sigmamomentum} .
\end{eqnarray}
As in classical mechanics these two quantities are momentum density currents for each of the two parameters, so we will refer to them as $\mathcal{P}^\tau_\mu$ and $\mathcal{P}^\sigma_\mu$ respectively. Now, following the pattern of the point particle, we want to split the integrand into total derivatives and boundary terms. In this case there are two parameters, so instead of two terms under the integral we have four.
\begin{equation}\label{variint}
\delta S = 0 = \int_{\tau_i}^{\tau_f} d\tau \int_0^\pi d\sigma \left[\frac{\partial(\delta X^\mu \mathcal{P_\mu^\tau})}{\partial \tau} + \frac{\partial (\delta X^\mu \mathcal{P}^\sigma_\mu)}{\partial \sigma}  - \delta X^\mu\frac{\partial \mathcal{P}_\mu^\tau}{\partial \tau} - \delta X^\mu\frac{\partial \mathcal{P}_\mu^\sigma}{\partial \sigma}\right]
\end{equation}
The first term is the equivalent of the first term in equation \ref{pointseparated}, and vanishes automatically for the same reason; we can simply specify the initial and final configurations of the string in order to make it vanish. We can group the third and fourth terms together due to their mutual factor of $\delta X^\mu$, and since together they must vanish for all such variations we have that
\begin{equation}\label{stringeom}
\frac{\partial \mathcal{P}_\mu^\tau}{\partial \tau} + \frac{\partial \mathcal{P}_\mu^\sigma}{\partial \sigma} = 0,
\end{equation}
our equation of motion for a relativistic string. This expression for the equations of motion looks relatively simple until you recall the definitions of the momentum density currents \ref{taumomentum} and \ref{sigmamomentum}.

 The remaining term in \ref{variint} corresponds to the string endpoints\footnote{Or continuity conditions in the case of the closed string.}. Due to the summation implied by the pair of '$\mu$'s, there are actually $2d$ boundary conditions contained within this term, one per end of the string per spacetime dimension. The simplest way to ensure the boundary conditions are satisfied is to say that $\mathcal{P}^\sigma_\mu$ must vanish at the end points for all $\mu$.  
\subsection{The Static Gauge}\label{staticgauge}
An advantage of reparameterization invariance is that it allows us to freely choose a parameterization to work with, simplifying the manipulation required significantly. In this case, it isn't necessary to entirely fix the parameterization; it suffices to specify the $\tau$ parameterization. We choose to associate lines of constant $\tau$ to the time coordinate $X^0$ in some choice of Lorentz frame. The sigma parameterization is not specified except that it takes the values $0$ and $\pi$ at the string endpoints. This choice is called the \emph{static gauge}.

In this gauge, the coordinates $X^\mu$ are slightly less free; the first coordinate naturally takes the value $ct$, and the remaining $d-1$ coordinates become a spacial vector. With this we obtain the expressions
\begin{eqnarray}
X^\mu(t,\sigma) & = & (ct, \mathbf{X}(t,\sigma)), \\
\frac{\partial X^\mu}{\partial \sigma} & = & \left(0,\frac{\partial \mathbf{X}}{\partial\sigma}\right), \\
\frac{\partial X^\mu}{\partial \tau} & = & \left(c, \frac{\partial \mathbf{X}}{\partial \tau}\right) .
\end{eqnarray}
We can also calculate the terms under the square root in the action in the static gauge. We obtain
\begin{eqnarray}
\dot{X}.X' & = & \frac{\partial \mathbf{X}}{\partial t}.\frac{\partial \mathbf{X}}{\partial \sigma}\label{foo1} \\
(\dot{X})^2 & =& -c^2 + \left(\frac{\partial \mathbf{X}}{\partial t}\right)^2 \\
(X')^2 & = & \left(\frac{\partial \mathbf{X}}{\partial \sigma}\right)^2\label{foo2}
\end{eqnarray}
\subsection{Transverse Velocity Action}
In order to make the equations of motion easier to interpret, we can re-express them in terms of the transverse velocity. The longitudinal velocity turns out not to be physically significant, due to the parameterization independence of the worldsheet. In any Lorentz frame it is possible to find a snapshot of the string corresponding to a fixed value of $\tau$. However, if in that Lorentz frame you were to take two snapshots of the string with an infinitesimal time separation $dt$ there would be no way to say that one 'point' of the string at the first time corresponded to an equivalent 'point' on the string in the second snapshot. This is a fundamental property from reparameterization invariance, which means that any longitudinal motion could not be significant.

The transverse velocity, however, is dynamically significant. The easiest way to visualise the transverse velocity is to consider the \emph{spatial} surface swept out by the string moving through space. In effect, this is the surface created by taking snapshots of the string at all times (in your frame), and sticking them all together. If you do this, the transverse velocity at a point on the string is the vector orthogonal to the string, and tangent to this surface.

We work in the static gauge with time coordinate $t$, and call the distance along the string in the snapshots $s$. The velocity of the string would naturally be given by $\partial\mathbf{X}/\partial t$, and we want to remove the component of this velocity along the string's length. Since $s$ is defined to be the distance along the string, the vector $\partial \mathbf{X} / \partial s$ has unit length, and points along the string, so we can find the longitudinal component by simply taking the dot product. Using this, we can find an expression for the transverse velocity,
\begin{equation}
\mathbf{v}_\perp = \frac{\partial \mathbf{X}}{\partial t} - \left(\frac{\partial \mathbf{X}}{\partial t}.\frac{\partial \mathbf{X}}{\partial s}\right) \frac{\partial \mathbf{X}}{\partial s}.
\end{equation}

Now, using equations \ref{foo1}-\ref{foo2}, we reconstruct the string action.
The argument in the square root becomes
\begin{eqnarray*}
(\dot{X}.X')^2 - (\dot{X})^2(X')^2& = &\left(\frac{\partial \mathbf{X}}{\partial t}.\frac{\partial \mathbf{X}}{\partial \sigma}\right)^2 - \left(c^2-\left(\frac{\partial \mathbf{X}}{\partial t}\right)^2\right)\left(\frac{\partial \mathbf{X}}{\partial \sigma}\right)^2 \\
& = & \left[ c^2 + \left(\frac{\partial \mathbf{X}}{\partial t}.\frac{\partial \mathbf{X}}{\partial s}\right) -\left(\frac{\partial \mathbf{X}}{\partial t}\right)^2\right]\left(\frac{ds}{d\sigma}\right)^2
\end{eqnarray*}
where for the second line we have used  
\begin{equation}
\frac{\partial \mathbf{X}}{\partial \sigma} = \frac{\partial \mathbf{X}}{\partial s} \frac{ds}{d\sigma},
\end{equation}
which is a simple application of the chain rule, and the fact that $\partial \mathbf{X}/\partial s$ is a unit vector.
Now, the second and third terms inside the square bracket turn out to be equal to $v_\perp^2$, as
\begin{eqnarray*}
v_\perp^2& =& \left(\frac{\partial \mathbf{X}}{\partial t} - \left(\frac{\partial \mathbf{X}}{\partial t}.\frac{\partial \mathbf{X}}{\partial s}\right) \frac{\partial \mathbf{X}}{\partial s}\right)\left(\frac{\partial \mathbf{X}}{\partial t} - \left(\frac{\partial \mathbf{X}}{\partial t}.\frac{\partial \mathbf{X}}{\partial s}\right) \frac{\partial \mathbf{X}}{\partial s}\right) \\ 
& = & \left(\frac{\partial \mathbf{X}}{\partial t}\right)^2 - 2\left(\frac{\partial \mathbf{X}}{\partial t}.\frac{\partial \mathbf{X}}{\partial s}\right)^2 + \left(\frac{\partial \mathbf{X}}{\partial t}.\frac{\partial \mathbf{X}}{\partial s}\right)^2\left(\frac{\partial \mathbf{X}}{\partial s} \right)^2 \\
& = &  \left(\frac{\partial \mathbf{X}}{\partial t}\right)^2 -\left(\frac{\partial \mathbf{X}}{\partial t}.\frac{\partial \mathbf{X}}{\partial s}\right)^2 \left(2 - \left(\frac{\partial \mathbf{X}}{\partial s} \right)^2\right) \\
& = &  \left(\frac{\partial \mathbf{X}}{\partial t}\right)^2 -\left(\frac{\partial \mathbf{X}}{\partial t}.\frac{\partial \mathbf{X}}{\partial s}\right)^2. 
\end{eqnarray*}
This allows us to write the square root as
\begin{equation}
c\frac{ds}{d\sigma}\sqrt{1-\frac{v_\perp^2}{c^2}},
\end{equation}
and the $c$ outside the square root cancels with the factor of $1/c$ to give
\begin{equation}
S = -T_0\int dt \int_0^\pi d\sigma \frac{ds}{d\sigma}\sqrt{1-\frac{v_\perp}{c^2}}.
\end{equation}
While we could choose to cancel the $d\sigma$ in this expression, it is useful to keep $\sigma$ around as it has definite bounds. $\sigma$ will always vary between 0 and $\pi$, while the length of the string may be time dependent. However, it is useful to briefly express this in terms of the length parameter, and ignore the time integral in order to find the Lagrangian.
\begin{equation}
L = -T_0 \int ds \sqrt{1-\frac{v^2_\perp}{c^2}}
\end{equation}
Comparing this to equation \ref{pointlagrangian} we can see a nice comparison. In the point particle case the Lagrangian was the rest energy modified by a relativistic factor. In the case of the string, the total rest energy of the string is given by  
\begin{equation}
T_0\int ds,
\end{equation}
and in this case the relativistic factor varies along the string, hence the $\int d\sigma$. This affirms our choice of action as the natural extension of the point particle action.

There is one more result we can get from the transverse velocity action. Firstly, in this set up we can find a new expression for $\mathcal{P}^{\sigma\mu}$.By substituting in various terms and rearranging, equation \ref{sigmamomentum} becomes
\begin{equation}\label{erty}
\mathcal{P}^{\mu\sigma} = -\frac{T_0}{c}\frac{\left(\frac{\partial X^\nu}{\partial s}\frac{\partial X_\nu}{\partial t}\right)\dot{X}^\mu + \left(c^2 -\left(\frac{\partial X^\nu}{\partial t}\frac{\partial X_\nu}{\partial t}\right)^2\right)\frac{\partial X^\mu}{\partial s}}{\sqrt{1-\frac{v_\perp^2}{c^2}}}.
\end{equation}
We chose to satisfy the boundary conditions by requiring $\mathcal{P}^{\sigma\mu}$ to vanish at the end points. With this expression for $\mathcal{P}^{\sigma\mu}$ this means the numerator must vanish at the end points for all $\mu$. In particular, it must vanish for $\mu=0$. By our choice of direction $s$, $\partial X^0/\partial s = 0$, so the first term in the numerator of \ref{erty} must vanish independently of the second. As such
\begin{equation}
\frac{\partial X^\nu}{\partial s}\frac{\partial X_\nu}{\partial t} = 0.
\end{equation}
But the first derivative, $\partial X^\nu/\partial s$ is the fixed unit vector along the string. In order for the entire expression to vanish the motion of the endpoints, $\partial X^\nu/\partial t$ must be perpendicular to $\partial X^\nu/\partial s$.
For $\mu \ne 0$ the second term in the numerator of \ref{erty} must still vanish so we have
\begin{equation}
c^2 = \frac{\partial X^\nu}{\partial t}\frac{\partial X_\nu}{\partial t} = |v^2|
\end{equation}
so the endpoints of the string move perpendicular to the string at the speed of light.
\subsection{String Momentum}

In many cases in classical mechanics, the conservation of momentum of free systems allows us to deduce important features of the system dynamics. As such, it would be extremely useful to find some kind of conserved momentum for a freely moving string. We can do this using the technique from Lagrangian mechanics for finding a conserved charge. In this case, the symmetry that generates this charge is invariance under spacetime translations. Recall the Lagrangian density
\begin{equation}
\mathcal{L}(\dot{X}, X') = \frac{-T_0}{c}\sqrt{(\dot{X}.X')^2 - (\dot{X})^2(X'^2)},
\end{equation}
and note that this has no $X$ dependence, it only depends on its derivatives, so clearly this Lagrangian density has spacetime translation invariance. We can write down an arbitrary translation as a variation as
\begin{equation}
\delta X^\mu(\tau, \sigma) = a^\mu,
\end{equation}
where $a^\mu$ is the translation. To be fully explicit, we are claiming that if the entire worldsheet is translated in spacetime by a vector $a^\mu$, the Lagrangian density is conserved. To find the conserved current $j_\mu^\alpha$ corresponding to this variation, we use the standard result \cite{peskin}
\begin{equation}
a^\mu j_\mu^\alpha = \frac{\partial \mathcal{L}}{\partial (\partial_\alpha X^\mu)} \delta X_\mu
\end{equation}
where $\alpha$ indexes the parameters of the worldsheet (in this case $\sigma, \tau$). In this case, this equation becomes
\begin{equation}
a^\mu j_\mu^\alpha = \frac{\partial \mathcal{L}}{\partial (\partial_\alpha X^\mu)} a^\mu.
\end{equation}
Cancelling the factors of $a^\mu$, and substituting our parameters in place of the '$\alpha$'s, this looks more familiar, 
\begin{equation}
j^\tau_\mu = \frac{\partial \mathcal{L}}{\partial \dot{X}} = \mathcal{P}_\mu^\tau, \quad j^\sigma_\mu = \frac{\partial \mathcal{L}}{\partial X'} = \mathcal{P}_\mu^\sigma.
\end{equation}
Then, the current conservation condition, $\partial_\alpha j_\mu^\alpha = 0$ is simply the string equations of motion \ref{stringeom}
\begin{equation}\tag{\ref{stringeom}}
\frac{\partial \mathcal{P}_\mu^\tau}{\partial \tau} + \frac{\partial \mathcal{P}_\mu^\sigma}{\partial \sigma} = 0,
\end{equation}
The conserved charge associated with a conserved current is given by the integral over space of the $j_\mu^0$ components, in this case $j_\mu^\tau$.
\begin{equation}\label{momentumconstanttau}
p^\mu = \int_0^\pi\mathcal{P}_\mu^\tau(\tau, \sigma)d\sigma,
\end{equation}
for fixed values of $\tau$. This is our conserved momentum, 
\begin{equation}
\frac{dp_\mu}{d\tau} = 0.
\end{equation}
Notice that this is a strong statement, the derivative is with respect to $\tau$, as opposed to some frame's $t$. This means that this quantity is conserved with respect to $\tau$ regardless of the choice of parameterization, as long as lines of constant $\tau$ are spacelike. In particular, back in our static gauge, we do indeed have the property 
\begin{equation}
\frac{dp_\mu}{dt} = 0.
\end{equation}

We can in fact find a slightly more general definition of the momentum $p^\mu$. For any curve $C$ that starts at the $\sigma = 0$ boundary of the worldsheet, and ends at the $\sigma = \pi$ boundary, 
\begin{equation}\label{generalmomentum}
p^\mu(C) = \int_C \mathcal{P}_\mu^\tau d\sigma - \int_C \mathcal{P}_\mu^\sigma d\tau.
\end{equation}
Then this quantity is the fixed for all choices of $C$. This is presented without proof, although it is easy to see how this collapses into the previous definition \ref{momentumconstanttau} for fixed values of $\tau$, as in that case $d\tau = 0$.
\subsection{Lorentz Charges}\label{lorentzcharges}
The momentum in the previous section was found by making use of the invariance of the action under spacetime translations. However, thanks to its careful construction, our action has an additional invariance: Lorentz invariance. It is instructive to consider the charges that we obtain from this invariance.

A spacetime translation could be described by a Lorentz vector $a^\mu$. An arbitrary Lorentz transformation is characterised by a type $(0,2)$ anti-symmetric tensor $a^{\mu\nu}$, and its effect is given by
\begin{equation}
\delta X^\mu(\tau,\sigma) = a^{\mu\nu}X_\nu.
\end{equation}
If we follow the same procedure as for the momentum to find conserved currents, but this time using the Lorentz transformation, we obtain a set of conserved currents given by
\begin{equation}\label{angularmomentum}
\mathcal{M}_{\mu \nu}^\alpha = X_\mu \mathcal{P}_\nu^\alpha - X_\nu\mathcal{P}_\mu^\alpha,
\end{equation}
 with $\alpha$ taking the values $\tau, \sigma$.
The set of conserved charges are given by
\begin{equation}
M_{\mu\nu} = \int_C \mathcal{M}_{\mu\nu}^\tau d\sigma + \int_C \mathcal{M}_{\mu\nu}^\sigma d\tau.
\end{equation}
From the form of equation \ref{angularmomentum}, we can see that these charges are in fact angular momentum charges. Moreover, this means that the conservation of angular momentum generates Lorentz invariance. This fact will become important later when we quantize the string.

We have now obtained the string equations of motion, a conserved momentum and set of conserved angular momenta, and have a certain amount of intuition gained from using the static gauge. However, any particular solutions of the equations of motion are still far too complicated to comfortably work with. What we need is a new choice of gauge that simplifies the equations of motion so that we can find solutions of the equations of motion that we can try to quantize.

\subsection{Gauge Fixing}
In section \ref{staticgauge} we chose to associate lines of constant $\tau$ to a single point in time in some Lorentz frame. Motivated by this, we can imagine constructing more general gauges where we associate $\tau$ to linear combinations of spacetime coordinates, and some scaling parameter. We can write this idea as an equation, 
\begin{equation}
\lambda \tau = n_\mu x^\mu. 
\end{equation} 
The static gauge is just a special case of this, with $n_\mu = (1, 0, \dots, 0)$. In the process of selecting a gauge we are naturally going to lose Lorentz covariance. By selecting the vector $n_\mu$ we are choosing a preferred direction to fix $\tau$, and no such vector is invariant under all Lorentz transformations. This is fine for now, but is an issue that must be addressed later. By convention, we modify our gauge condition slightly. As the momentum $p^\mu$ given by equation \ref{generalmomentum} is a conserved vector, the scalar quantity $(p^\mu\,n_\mu)$ is a constant scalar quantity, and we choose to write
\begin{equation}
n_\mu X^\mu (\tau, \sigma) = \bar{\lambda}(p^\mu\,n_\mu)\tau.
\end{equation}

Now that we have a $\tau$ parameterization, we need to fix the $\sigma$ parameterization. We can do this by choosing a parameterization such that $n_\mu\mathcal{P}^{\tau \mu}$ is constant over the strings for a given value of $\tau$, i.e.
\begin{equation}\label{sigmafix}
n_\mu \mathcal{P}^{\sigma \mu}(\tau, \sigma) = a(\tau).
\end{equation}
Actually, it turns out that $a$ cannot depend on $\tau$. Integrating both sides of equation \ref{sigmafix} gives 
\begin{eqnarray*}
\int_0^\pi n_\mu \mathcal{P}^{\sigma \mu}(\tau, \sigma) d\sigma & = & \int_0^\pi a(\tau) d\sigma\\
n_\mu\, p^\mu &= &\pi a(\tau),
\end{eqnarray*}
so
\begin{equation}\label{ndotp}
a(\tau) = \frac{n_\mu\,p^\mu}{\pi}, 
\end{equation}
and we can see that $a(\tau)$ has no $\tau$ dependence, and in fact its value is already fixed.

We can use this to find an important property of string motion. Taking the scalar product of our equations of motion \ref{stringeom} and our choice of gauge $n_\mu$, we obtain
\begin{eqnarray*}
n_\mu\frac{\partial \mathcal{P}^{\tau \mu}}{\partial \tau} + n_\mu\frac{\partial \mathcal{P}^{\sigma \mu}}{\partial \sigma} & = &\\
\frac{\partial (n_\mu \mathcal{P}^{\tau \mu})}{\partial \tau} + \frac{\partial(n_\mu \mathcal{P}^{\sigma \mu})}{\partial \sigma} & = & 0
\end{eqnarray*}
The first term here vanishes, as we just proved that $n_\mu \mathcal{P}^{\tau \mu}$ has no $\tau$ dependence. This leaves us with
\begin{equation}\label{foo3}
\frac{\partial (n_\mu \mathcal{P}^{\sigma \mu})}{\partial \sigma} = 0. 
\end{equation}
For open strings\footnote{For closed strings this is more complicated, as there is no preferred point on the string that we can choose to fix. You achieve the same result, but the derivation is not provided here.}, we can choose to have $n_\mu \mathcal{P}^{\sigma \mu} = 0$ at the string end points, and so equation \ref{foo3} allows us to conclude
\begin{equation}
n_\mu \mathcal{P}^{\sigma \mu} = 0.
\end{equation}

We now have a full parameterization for our arbitrary gauge $n_\mu$.
\begin{equation}\label{ngauge}
n_\mu \, X^\mu(\tau, \sigma)  =  \frac{2}{T_0 \hbar c} n_\mu  \mathcal{P}^{\tau \mu} \tau 
\end{equation}
With this framework in place, we can attempt to construct the equations of motion and constraints in a more palatable way. First, we examine equation \ref{sigmamomentum}, which is the full expression for $\mathcal{P}^\sigma_\mu$. Taking the scalar product with $n_\mu$ gives
\begin{equation}{\label{dirmom}}
n_\mu \mathcal{P}^{\sigma \mu} = \frac{-T_0}{c} \left(\frac{(\dot{X}_\mu X'^\mu) \frac{\partial (n_\nu X^\mu)}{\partial \tau} - (X')^2\frac{\partial (n_\nu X'^\nu)}{\partial \sigma}}{\sqrt{(\dot{X}.X') - (\dot{X})^2(X')^2}}\right).
\end{equation}
Then, using \ref{foo3}, we see that the second term in the numerator vanishes. Similarly, $\partial (n_\mu \mathcal{P}^{\tau \mu})/\partial \tau$ is a constant, so in order that equation \ref{dirmom} vanishes, $\dot{X}.X'$ must be equal to $0$. Using this, we can simplify equation \ref{taumomentum}, the expression for $\mathcal{P}^{\tau \mu}$.
\begin{equation}\label{foo5}
\mathcal{P}^{\tau \mu} = \frac{T_0}{c}\frac{(X')^2 \dot{X}_\mu}{\sqrt{-(\dot{X})^2(X')^2}}
\end{equation}
Then using property \ref{ndotp} of our gauge we find
\begin{equation}
1 = \frac{X'^2}{\sqrt{-(\dot{X})^2(X')^2}}
\end{equation}
and this can be rearranged into 
\begin{equation}\label{pluscondition}
\dot{X}^2 + X'^2 = 0.
\end{equation}
Conveniently, this condition, along with the aforementioned property that $\dot{X}.X'=0$ can be combined into the single statement
\begin{equation}\label{constraint}
(\dot{X}\pm X')^2 = 0.
\end{equation}
This condition is enough to finally put the equations of motion into an understandable form. Picking up from equation \ref{foo5}, we can simplify the square root using \ref{pluscondition}, $\sqrt{-\dot{X}^2 X'^2} = \sqrt{X'^4} = X'^2$, as we have already guaranteed the positivity of the square root. Dividing through by the $X'^2$ from the numerator gives 
\begin{equation}\label{taumomentumeom}
\mathcal{P}^{\tau \mu} = \frac{T_0}{c}\dot{X}^\mu.
\end{equation}
Similarly, we can now write
\begin{equation}
\mathcal{P}^{\sigma \mu} = \frac{-T_0}{c} X'^\mu.
\end{equation}
Recalling the equations of motion \ref{stringeom}, we find that in this parameterization
\begin{equation}
\ddot{X}^\mu -X''^\mu = 0
\end{equation}
which is a case of the wave equation. The corresponding boundary conditions are
\begin{equation}
\left.X'^\mu \right|_{\sigma = 0} = 0, \quad\left. X'^\mu \right|_{\sigma = \pi} = 0. 
\end{equation}

\subsection{Wave Equation General Solution}\label{wegs}
As the solution of the wave equation is relatively straightforward, we will not discuss it in detail. For a full formal treatment see Strauss\cite{strauss}.
Since the quantity $1/(2\pi T_0 \hbar c)$ appears very frequently, it is useful to abbreviate this to $\alpha'$. This quantity also has a physical significance, which makes it an appropriate choice of constant.
The full expansion of the general solution is given by
\begin{equation}
X^\mu (\tau, \sigma) = x_0^\mu + 2\alpha' p^\mu \tau - i\sqrt{2\alpha'}\sum_{n=0}^\infty \frac{1}{\sqrt{n}}\left(\bar{a}_n^\mu e^{in\tau} - a_n^\mu e^{-in\tau}\right) \cos n\sigma,
\end{equation}
where the $a_n^\mu$ are constants, and an over-bar means complex conjugation.

The terms outside the sum correspond to the centre of mass motion of the string, and the oscillations are described by the sum.

This expression can in fact be generalised further. We will find a new set of constants, $\alpha_n^\mu$ to replace the $a_n^\mu$ . Since we are summing two oppositely signed exponentials over the natural numbers, by careful definition we could turn the two summed terms into a single sum over the integers. While doing this, we will also rescale slightly to remove the square root in the sum. Firstly, for $n>1$, we choose to define 
\begin{equation}\label{atoalpha1}
\alpha_n^\mu := \sqrt{n} a_n^\mu.
\end{equation} 
This is simply a rescaling. Then, to reduce to a single sum we define (again for $n>1$)
\begin{equation}\label{atoalpha2}
\alpha_{-n}^\mu := \sqrt{n}\bar{a}_n^\mu. 
\end{equation}
Using these we can rewrite the summation as
\begin{equation}
\sum_{n \ne 0} \frac{1}{n}\alpha_n^\mu e^{in\tau}\cos n\sigma
\end{equation}
since $\cos$ is an even function.

If we choose to define
\begin{equation}\label{pmudefn}
a_0^\mu := \sqrt{2\alpha'}p^\mu,
\end{equation}
then we can write
\begin{equation}\label{modeexp}
X^\mu (\tau, \sigma) = x_0^\mu + \sqrt{2\alpha'} \alpha_0^\mu \tau + i\sqrt{2\alpha'}\sum_{n=0}^\infty \frac{1}{n} a_n^\mu e^{-in\tau} \cos n\sigma.
\end{equation}
With this as our expansion, we can calculate the derivatives $\dot{X}^\mu$ and $X'^\mu$.
\begin{eqnarray}
\dot{X}^\mu & = &\sqrt{2\alpha'}\sum_{n=-\infty}^\infty \alpha_n^\mu \cos n\sigma\, e^{-in\tau}\label{xdotdefn} \\
X'^\mu &=& -i\sqrt{2\alpha'}\sum_{n=-\infty}^\infty \alpha_n^\mu \sin n\sigma\, e^{-in\tau}\label{xdashdefn}
\end{eqnarray}
Using these we can check that this expansion satisfies our equations of motion.
\begin{eqnarray*}
0 &= &\ddot{X}^\mu -X''^\mu\\
& = & \sqrt{2\alpha'} \sum_{n=-\infty}^\infty \alpha_n^\mu \cos n\sigma\frac{d}{d\tau}( e^{-in\tau}) - i\sqrt{2\alpha'}\sum_{n=-\infty}^\infty \alpha_n^\mu \frac{d}{d\sigma}(\cos n\sigma)e^{-in\tau}\\
& =& -i\sqrt{2\alpha'} \sum_{n=-\infty}^\infty n\,\alpha_n^\mu \cos n\sigma \, e^{-in\tau} + i\sqrt{2\alpha'}\sum_{n=-\infty}^\infty n\,\alpha_n^\mu \cos n\sigma \, e^{-in\tau}\\
& = & 0
\end{eqnarray*}
We could similarly use these definitions to check the constraint equations are satisfied, although we will not do so here.
\subsection{The Light Cone Gauge}
In the previous section, the choice of gauge was very general, but here it is useful to specify it more precisely. We will choose 
\begin{equation}
n^\mu = (1/\sqrt{2},\quad 1/\sqrt{2},\quad 0, \dots, 0).
\end{equation}
Notationally, it is also useful to define, for an arbitrary vector $V^\mu$, the quantities $V^+$ and $V^-$ 
\begin{equation}
V^+ = \frac{V^0 + V^1}{\sqrt{2}}, \quad V^- = \frac{V^0 - V^1}{\sqrt{2}}. 
\end{equation}
With these quantities, we can write the spacetime interval, $ds^2$ as
\begin{equation}
-ds^2 = -2dV^+dV^- + (dV^2) + \dots + (dV^{d-1})^2,
\end{equation}
and the Lorentz scalar product of two vectors $V^\mu$, $U^\mu$ is 
\begin{equation}
V^\mu_\mu = -V^+ U_+ - V^-U_- + V^2U_2 + \dots + V^{d-1}U_{d-1}. 
\end{equation}
This choice of gauge is called the \emph{Light-cone gauge}. We will use this gauge to show that all of the dynamics is contained within the transverse coordinates, $X^2, \dots, X^{d-1}$. Call these transverse coordinates $X^i$, where $i$ runs from $2$ up to $d$. Thus, the vector $X^\mu$ is written as $(X^+, X^-, \mathbf{X}^i)$  
Using this notation and our choice of gauge we have 
\begin{eqnarray}
X^+(\tau, \sigma) &=& 2\alpha' p^+ \tau\label{xplusdefn} \\
X^-(\tau, \sigma) &=& 2\alpha' p^- \tau \\
p^+ & = & \pi \mathcal{P}^{\tau +}\\
p^- & = & \pi \mathcal{P}^{\tau -}. 
\end{eqnarray} 
In this gauge, the constraint equation \ref{constraint} becomes
\begin{eqnarray*}
(\dot{X} \pm X')^2& =&
-2(\dot{X}^+ \pm X'^+)(\dot{X}^- \pm X'^-) + (\dot{\mathbf{X}}^i \pm \mathbf{X}'^i)^2\\ & = & 0 .
\end{eqnarray*}
Using equation \ref{xplusdefn}, we can see that $\dot{X}^+ = 2\alpha'p^+$ and $X'^+=0$, so this constraint can be rewritten as
\begin{equation}
0 = -4\alpha'p^+(\dot{X}^- \pm X'^-) +  (\dot{\mathbf{X}}^i \pm \mathbf{X}'^i)^2,
\end{equation}
or, assuming $p^+ \ne 0$
\begin{equation}\label{pmderivatives}
(\dot{X}^- \pm X'^-) = \frac{ (\dot{\mathbf{X}}^i \pm \mathbf{X}'^i)^2}{4\alpha'p^+}.
\end{equation}
In the case where $p^+ = 0$, or equivalently $p^0 + p^1 = 0$ (which would mean a massless particle travelling at speed $c$ in the negative $x^1$ direction), the following will not apply, but as this case is so rare it is mostly safe to ignore it.

This pair of equations can be solved for $\dot{X}^-$ and $X'^-$,
\begin{eqnarray}
\dot{X}^- &=& \frac{(\dot{\mathbf{X}}^i + \mathbf{X}'^i)^2 + (\dot{\mathbf{X}}^i - \mathbf{X}'^i)^2}{2\alpha'p^+}\\
&=& \frac{(\dot{\mathbf{X}}^i)^2 + (\mathbf{X}'^i)^2}{\alpha'p^+}\label{xdotminus}\\
X'^- &=& \frac{(\dot{\mathbf{X}}^i + \mathbf{X}'^i)^2 - (\dot{\mathbf{X}}^i - \mathbf{X}'^i)^2}{2\alpha'p^+}\\
&=& \frac{2\dot{\mathbf{X}}^i.\mathbf{X}'^i}{\alpha'p^+}.
\end{eqnarray}
In order to get a closed form expression for $X^-$ given the derivatives, all we need to calculate it is a value of $X^-$ at for some coordinates $\tau_0$, $\sigma_0$, and then an expression for $X^-$ at some other point can be found by
\begin{eqnarray*}
X^-(\tau_1, \sigma_1) &=& X^-(\tau_0, \sigma_0) + \int_C dX^-\\
&=& X^-(\tau_0, \sigma_0) + \int_C \left(\frac{\partial X^-}{\partial \tau} d\tau +  \frac{\partial X^-}{\partial \sigma}d\sigma \right)
\end{eqnarray*}
where the curve C connects the two points. But we saw from the expressions for $\dot{X}^-$ and $X'^-$ that they only depend on the transverse coordinates $\mathbf{X}^i$, the $+$ component of momentum, and a single value of $X^-$, which we will call $x^-_0$. Noting that our expression for $X^+$, equation \ref{xplusdefn}, also only depends on $p^+$, we see that dynamics of the string only depend on these quantities, 
\begin{equation}
\mathbf{X}^i(\tau, \sigma), \quad p^+, \quad x^-_0.
\end{equation}
\subsection{Light-Cone Equations of Motion}\label{lceom}
Using our mode expansion of the general solution, equation \ref{modeexp}, the transverse coordinates $X^i$ are given by
\begin{equation}\label{imodeexp}
X^i (\tau, \sigma) = x_0^i + \sqrt{2\alpha'} a_0^i \tau + i\sqrt{2\alpha'}\sum_{n=0}^\infty \frac{1}{n} \alpha_n^i e^{-in\tau} \cos n\sigma.
\end{equation}
Also, from equations \ref{xplusdefn} and \ref{pmudefn} we have
\begin{equation}
X^+(\tau, \sigma) = 2\alpha'p^+\tau. \label{xplusfoo}
\end{equation}
This is in fact the mode expansion for the case where the centre-of-mass position $x^+_0$ and all $X^+$ oscillations vanish.

Finding the $X^-$ coordinates is a little more involved. $X^-$ is a linear combination of $X^0$ and $X^1$, which are both solutions of the equations of motion, so $X^-$ is also a solution of the equations of motion, so we can still use the mode expansion \ref{modeexp}. $x_0^-$ is one of the dynamical components, so that term of the expansion can be left untouched. What we are really seeking is expressions for the minus oscillators, $\alpha_n^-$. 

Note that we can use the derivatives that we calculated in equations \ref{xdotdefn} and \ref{xdashdefn} to write
\begin{equation}\label{sumderivatives}
\dot{X}^\mu \pm X'^\mu = \sqrt{2\alpha'}\sum_{n \in \mathbb{Z}}\alpha_n^\mu e^{-in(\tau \pm \sigma)},
\end{equation}
since $\cos \theta + i \sin \theta = e^{i\sin\theta}$. In particular the minus $X^-$ derivatives satisfy this, and using equation \ref{pmderivatives} we have
\begin{eqnarray}
\sqrt{2\alpha'}\sum_{n \in \mathbb{Z}}\alpha_n^- e^{-in(\tau \pm \sigma)} &=& \dot{X}^- \pm X^-\nonumber\\
&=& \frac{(\mathbf{X}^i \pm \mathbf{X}'^i)^2}{4\alpha'p^+}\label{foo6}.
\end{eqnarray}
equation \ref{sumderivatives} also holds for the $\mathbf{X}^i$ derivatives, so 
\begin{equation}
\dot{X}^i \pm X^i = \sqrt{2\alpha'}\sum_{n \in \mathbb{Z}}\alpha_n^i e^{-in(\tau \pm \sigma)},
\end{equation}
allowing us to continue \ref{foo6}
\begin{eqnarray*}
&=& \frac{1}{2p^+}\left(\sum_{a \in \mathbb{Z}}\alpha_a^i e^{-ia(\tau \pm \sigma)}\right)\left(\sum_{b \in \mathbb{Z}}\alpha_b^i e^{-ib(\tau \pm \sigma)}\right)\\
&=& \frac{1}{2p^+}\sum_{a \in \mathbb{Z}}\sum_{b \in \mathbb{Z}}\alpha_a^i\alpha_b^ie^{-i(a+b)(\tau \pm \sigma)}.
\end{eqnarray*}
Now, setting $n=(a+b)$ we can shift one of the sums to get
\begin{equation}
= \frac{1}{2p^+}\sum_{n \in \mathbb{Z}}\sum_{a \in \mathbb{Z}}\alpha_a^i\alpha_{n-a}^ie^{-in(\tau \pm \sigma)}.
\end{equation}
Convenient addition of brackets gives us our result,
\begin{eqnarray}
\sqrt{2\alpha'}\sum_{n \in \mathbb{Z}} \alpha^-_n e^{in(\tau \pm \sigma)} &=& \frac{1}{2p^+}\sum_{n \in \mathbb{Z}}\left(\sum_{a \in \mathbb{Z}} \alpha_a^i\alpha_{n-a}^i\right)e^{in(\tau \pm \sigma)}\\
\alpha^-_n&=& \frac{1}{2p^+\sqrt{2\alpha'}}\sum_{a \in \mathbb{Z}}\alpha_{n-a}\alpha_a\label{minusoscillators}.
\end{eqnarray}
Finally, we have a complete explicit solution for the equations of motion in the light-cone gauge.
\begin{eqnarray}
X^+(\tau, \sigma) &=& 2\alpha'p^+ \tau \\
X^-(\tau, \sigma) &=& x_0^- + \sqrt{2\alpha'} \alpha_0^- \tau + i\sqrt{2\alpha'}\sum_{n=0}^\infty \frac{1}{n} \alpha_n^- e^{-in\tau} \cos n\sigma\\
X^i (\tau, \sigma) &=& x_0^i + \sqrt{2\alpha'} \alpha_0^i \tau + i\sqrt{2\alpha'}\sum_{n=0}^\infty \frac{1}{n} \alpha_n^i e^{-in\tau} \cos n\sigma\\
\text{where} \,\, \alpha^-_n&=& \frac{1}{2p^+\sqrt{2\alpha'}}\sum_{a \in \mathbb{Z}}\sum_{i=2}^{d-1}\alpha^i_{n-a}\alpha^i_a.
\end{eqnarray}

With these four expressions, a possible motion can be found by choosing arbitrary values for $p^+, x_0^-$, and fixing $X^i$ by choosing values for $x_0^i$ and $\alpha_n^i$ for all $n \in \mathbb{Z}$ and for all $2\leq i\leq d-1$.
\section{Quantizing the Relativistic Point Particle}\label{secpoint}
\subsection{Point Particle Equations of Motion in the Light-Cone Gauge}
Back in section \ref{sectionpoint} we found a Lorentz covariant action for a point particle, and then used it to find the equations of motion. We found that the action was proportional to the proper length along the string, and parameterizing the world line by proper time gave us the expression
\begin{equation}
S = -mc\int_{\tau_i}^{\tau_f} \sqrt{-\eta_{\mu \nu} \frac{dx^\mu}{d\tau} \frac{dx^\nu}{d\tau}}d\tau,
\end{equation}
which led to the equations of motion
\begin{equation}
\frac{dp_\mu}{d\tau} = 0,
\end{equation}
where
\begin{equation}
p_\mu := \frac{\partial L}{\partial \dot{x}^\mu} = \frac{m\dot{x}_\mu}{\sqrt{-\dot{x}_\nu\dot{x}^\nu}}.
\end{equation}
The corresponding constraint equation is
\begin{equation}\label{constraintpoint}
p^2 + m^2 = 0.
\end{equation}
We choose to use the light cone gauge, i.e. set the $x^+$ coordinate proportional to $\tau$. As with our choice of gauge for the string, we choose to have a factor of $p^+$ in the definition, and in order to correct the units we divide by $m^2$,
\begin{equation}\label{pointeom1}
x^+ := \frac{1}{m^2}p^+\tau.
\end{equation}
Substituting this into our definition of $p^\mu$ and cancelling the $p^+$ gives
\begin{eqnarray}
p^+ &=& \frac{m\dot{x}^+}{\sqrt{-x^\mu x_\mu}} = \frac{p^+}{m\sqrt{-x^\mu x_\mu}}\\
1 &=& \frac{1}{m\sqrt{-x^\mu x_\mu}} \\
x^\mu x_\mu &=& -\frac{1}{m^2}.
\end{eqnarray}
Thus
\begin{equation}\label{pmudefn2}
p_\mu = m^2 \dot{x}_\mu,
\end{equation}
and our equations of motion become
\begin{equation}
m^2\ddot{x}_\mu = 0.
\end{equation}
In the case of the particle, it is very easy to construct $p^-$ from $p^+$ and the transverse components. If we use our expression for the spacetime interval in light cone coordinates in equation \ref{constraintpoint}, we get
\begin{eqnarray}\label{pminusdefn}
p^2 + m^2 &=& -2p^+ p_- + p^i p_i + m^2 = 0\\
p_- &=& \frac{p^i p_i + m^2}{2p^+}.
\end{eqnarray}
Using equation \ref{pmudefn2} and integrating gives
\begin{eqnarray}\label{pointeom2}
\dot{x}^- &=& \frac{1}{m^2} p^- \nonumber\\
x^-(\tau) &=& \frac{1}{m^2}\int p^-d\tau\nonumber\\ 
&=& x_0^- + \frac{p^-}{m^2}\tau\nonumber\\
&=& x_0^- + \frac{p^i p_i + m^2}{2p^+m^2}\tau.
\end{eqnarray}
Similarly, 
\begin{equation}\label{pointeom3}
x^i(\tau) = x_0^i + \frac{p^-}{m^2}\tau.
\end{equation}
Together, equations \ref{pointeom1}, \ref{pointeom2} and \ref{pointeom3} give a complete set of equations for the motion of the point particle. From the right hand sides of these equations, we can see that the dynamically significant variables are, like in the case of the string,
\begin{equation}
p^+, \quad x_0^-, \quad x^i, \quad p^i.
\end{equation}
For the string we needed a full set of $\alpha_n^i$, but for a point particle you don't have the oscillatory motion, so it is only the values of $p^i$ that we need.
\subsection{Point Particle Quantum Operators}
Following Dirac quantization, we choose Heisenberg operators corresponding to each of our dynamical variables,
\begin{equation}
\hat{p}^+(\tau), \quad \hat{x}_0^-(\tau),\quad \hat{x}^i(\tau), \quad \hat{p}^i(\tau),
\end{equation} 
with canonical commutation relations (CCR)
\begin{equation}
[\hat{x}_0^-(\tau),\hat{p}^+(\tau)] = i\eta^{- +} = -i,\quad\quad 
[\hat{x}^i(\tau),\hat{p}^j(\tau)] = i\eta^{i j} = -i\delta^{i j}.
\end{equation}
All other commutators between these variables vanish.

We can use our definitions of the other, non dynamically significant variables to find their corresponding quantum operator equivalents as well. From equation \ref{pointeom1} we have
\begin{equation}\label{xplusheis}
\hat{x}^+(\tau) = \frac{1}{m^2}\hat{p}^+\tau.
\end{equation}
For $x^-$, it is useful to revert to the intermediary operator $p^-$
\begin{equation}
\hat{p}^- = \frac{\hat{p}^i\hat{p}_i + m^2}{2\hat{p}^+}.
\end{equation}
Then, using this, we can write the comparatively simple expression for $x^-$
\begin{equation}
\hat{x}^-(\tau) = \hat{x}_0^- + \frac{p^-}{m^2}\tau. 
\end{equation}
In order to find the commutation relations between these dependent operators, we substitute the CCR in their definitions. For example
\begin{eqnarray*}
[\hat{x}^+(\tau),\,\hat{x}^-(\tau)] &=& \left[\frac{\hat{p}^+}{m^2}\tau,\, \hat{x}_0^- + \frac{\hat{p}^i\hat{p}_i+m^2}{2\hat{p}^+m^2}\tau\right]\\
&=& \frac{1}{m^2}\left([\hat{p}^+\tau,\,\hat{x}^-_0] + \left[\hat{p}^+\tau,\, \frac{\hat{p}^i\hat{p}_i+m^2}{2\hat{p}^+m^2}\tau\right]\right)\\
&=& \frac{1}{m^2}\left(i\tau + 0\right)\\
&=& \frac{i\tau}{m^2}.
\end{eqnarray*} 
\subsection{Point Particle Hamiltonian}
The Hamiltonian of any quantum system is an operator that governs time evolution. In normal coordinates, this means that it is generated by $p^0$, the energy component of the momentum Lorentz vector. Since we are working in light-cone coordinates however, we will instead use light-cone energy ($p^-$) and light cone time ($x^+$)\footnote{We could equivalently choose $x^-$ to be the light cone time and $p^+$ the light cone energy. For a more thorough explanation of this, see Zwiebach\cite{zwiebach}.}. In these coordinates, light cone energy governs light cone time evolution, i.e.
\begin{equation}
\frac{\partial}{\partial x^+} \sim p^-.
\end{equation}
In order to find an operator that governs $\tau$ evolution, we use equation \ref{xplusheis} and rearrange to get
\begin{equation}
\tau = \frac{m^2\hat{x}^+}{\hat{p}^+}, 
\end{equation}
so we have that
\begin{equation}
\frac{\partial}{\partial \tau} \sim \frac{\hat{p}^+}{m^2}\hat{p}^-.
\end{equation}
This becomes our Heisenberg picture Hamiltonian,
\begin{eqnarray}
\hat{H}(\tau) &=& \frac{\hat{p}^+(\tau)}{m^2}\hat{p}^-(\tau)\nonumber\\
&=& \frac{\hat{p}^i(\tau)\hat{p}_i(\tau)+ m^2}{2m^2}\label{pointhamiltonian}.
\end{eqnarray}
When working in the Heisenberg picture, the time evolution of an arbitrary operator $\hat{A}(\tau)$ is given by
\begin{equation}
\frac{d\hat{A}(\tau)}{d\tau} = -i[\hat{A}(\tau), H(\tau)].
\end{equation}
We can use this to analyse the dynamics of our dynamically significant variables, and check that they evolve as expected using this Hamiltonian.

Demonstrating the conservation of $\hat{p}^+$, $\hat{x}_0^-$ and $\hat{p}^i$ is simple. Note that the Hamiltonian only depends on the transverse $\hat{p}^i$ operators, and by the CCR, $[\hat{p}^+, \hat{p}^i] = [\hat{p}^i,\hat{p}^j] = [\hat{x}_0^-,\hat{p}^i] = 0$. All operators commute with constants so the $m^2$ terms vanish as well. Explicitly for $\hat{p}^+$,
\begin{eqnarray*}
\frac{d \hat{p}^+(\tau)}{d\tau} &=& -i[\hat{p}^+(\tau),\hat{H}(\tau)]\\
&=& -i\left[\hat{p}^+(\tau),\frac{\hat{p}^i(\tau)\hat{p}_i(\tau) + m^2}{2m^2}\right]\\
&=& -\frac{i}{2m^2}\left[\hat{p}^+(\tau),\hat{p}^i(\tau)\hat{p}_i(\tau) \right]\\
&=& 0.
\end{eqnarray*}
The same argument holds for $\hat{x}_0^-$ and $\hat{p}^i$.

For $\hat{x}^i$ the manipulations are a little more involved.
\begin{eqnarray*}
\frac{d\hat{x}^i(\tau)}{d\tau} &=& -i\left[\hat{x}^i(\tau),\hat{H}(\tau)\right]\\ 
&=& -i\left[\hat{x}^i(\tau), \frac{\hat{p}^j(\tau)\hat{p}_j(\tau) + m^2}{2m^2}\right]\\
&=& -\frac{i}{2m^2}\left[\hat{x}^i(\tau),\eta_{k j}\hat{p}^k(\tau)\hat{p}^j(\tau)\right]\\
&=& -\frac{i\eta_{k j}}{2m^2}\left(\left[\hat{x}^i(\tau),\hat{p}^k(\tau)\right]\hat{p}^j(\tau) + \hat{p}^k(\tau)\left[\hat{x}^i(\tau),\hat{p}^j(\tau)\right]\right)\\
&=& -\frac{i}{2m^2}\left(i\eta_{k j}\eta^{i k}\hat{p}^j(\tau) + i\eta_{k j}\hat{p}^j(\tau)\eta^{i j}\right)\\
&=& \frac{1}{2m^2}(\hat{p}^i(\tau) + \hat{p}^i(\tau))\\
&=& \frac{\hat{p}^i(\tau)}{m^2}
\end{eqnarray*}
In the fourth equality, we have used the standard result of commutators that for operators $\hat{A}$, $\hat{B}$ and $\hat{C}$, $[\hat{A}\hat{B},\hat{C}]= [\hat{A},\hat{C}]\hat{B} + \hat{A}[\hat{B},\hat{C}]$.
 
This result fits naturally with our conventional ideas of momentum, as integrating gives the equations of motion
\begin{equation}
\hat{x}^i(\tau) = \hat{x}^i_0 + \frac{\hat{p}^i}{m^2}\tau,
\end{equation}
where $\hat{x}^i_0$ are constant operators of integration. Note that even in the Heisenberg picture, as $\hat{p}^i$, $\hat{x}_0^-$ and $\hat{p}^+$ are time independent, and $\hat{x}^i$ depend linearly on $\tau$ we can write
\begin{eqnarray}
\hat{p}^+(\tau) &=& \hat{p}^+\\
\hat{x}_0^-(\tau) &=& \hat{x_0^-}\\
\hat{x}^i(\tau) &=& \hat{x}_0^i + \frac{\hat{p}^i}{m^2}\tau\\
\hat{p}^i(\tau) &=& \hat{p}^i,
\end{eqnarray}
with all the operators on the right hand side being constant operators.
From here, we could easily derive the time evolution of our other, dependent operators, $\hat{x}^+$, $\hat{x}^-$ and $\hat{p}^-$, to see that the Hamiltonian works as desired. These results are omitted for brevity.
\subsection{Point Particle States and the Schr\"{o}dinger Equation}
To define the Hilbert space that our states will live in, we choose a set of commuting operators with as great a span as possible. Since the commutation requirement means we can only have one out of $\hat{p}^i$ and $\hat{x}^i$ for each $i$, and only one out of $\hat{x}_0^-$ and $\hat{p}^+$, we choose (for uniformity) to use the operators $p^+$ and the set of $p^i$. Thus, we write a state with eigenvalue $p^+$ for the $\hat{p}^+$ operator, and a vector of eigenvalues $\mathbf{p}^i$ for the set of $\hat{p}^i$ operators as
\begin{equation}
\ket{p^+, \mathbf{p}^i}.
\end{equation}
Then, by applying the Hamiltonian \ref{pointhamiltonian} to these states, we find the Schr\"{o}dinger equation,
\begin{eqnarray*}
i\hbar\frac{\partial}{\partial \tau}\ket{p^+,\mathbf{p}^i} &=& \hat{H}\ket{p^+,\mathbf{p}^i}\\
&=& \frac{\hat{p}^i(\tau)\hat{p}_i(\tau) + m^2}{2m^2}\ket{p^+,\mathbf{p}^i}.
\end{eqnarray*}
In fact, the most general state consists of a superposition of many such states. We can write this by defining a complex-valued function over the space of eigenvalues of our $p^+, \mathbf{p}^i$ operators and $tau$, denoted $\psi(q^+,\mathbf{q}^i,\tau)$. With this function defined, we can write our general state as
\begin{equation}
\ket{\psi,\tau} = \int dq^+ d\mathbf{q}^i \psi(q^+,\mathbf{q}^i,\tau)\ket{q^+,\mathbf{q}^i}.
\end{equation}
note that we can recover our first definition if we choose $\psi(p^+,\mathbf{p}^i,\tau) = \delta(q^+-q^+)\delta(\mathbf{q}^i-\mathbf{p}^i)$. With this more general set of states, the Schr\"{o}dinger equation instead becomes
\begin{equation}
i\frac{\partial}{\partial \tau} \psi(p^+,\mathbf{p}^i,\tau) = \frac{\hat{p}^i(\tau)\hat{p}_i(\tau) + m^2}{2m^2}\psi(p^+,\mathbf{p}^i,\tau).
\end{equation}

This gives us a good description on the quantum point particle. In the next section we will see how these same ideas can be applied to the quantum string.
\section{Quantum Relativistic Strings}\label{qrs}
Up to now, some indication has been given whenever there is a significant difference between the treatments for open and closed strings. However, from this point on the derivations diverge significantly, so we will only consider the case of open strings. 
\subsection{String Quantum Operators}
When we last discussed strings in section \ref{lceom}, we had a general solution for the string equation of motion obtained by fixing the following:
\begin{equation}
p^+,\quad x_0^-,\quad x_0^i,\quad \alpha_n^i\quad\quad\quad \forall\, n \in \mathbb{Z},\quad \forall\, 2 \leq i \leq d.
\end{equation}
Following from our results in section \ref{secpoint}, we expect that these parameters will become the time-independent operators of our quantum theory. However, we don't as yet have any intuition for how to treat the $\alpha_n^i$; in particular we don't know what to expect of its commutation relations. As such, for now we will use more familiar quantities, momenta and positions.

We can no longer treat the transverse positions or momenta as dependent variables, as they were given by their mode expansion using the $\alpha_n^i$s. These are now added to our set of independent operators. The (Heisenberg) operators we will use to quantize the string are
\begin{equation}
\hat{p}^+(\tau),\quad \hat{x}_0^-(\tau),\quad \hat{\mathcal{P}}^{\tau i}(\tau,\sigma),\quad \hat{X}^i(\tau,\sigma).
\end{equation}
Since all of these are positions or momenta, we can use the canonical equal-time commutation relations (ETCR) as an ansatz. We will set
\begin{eqnarray}
\left[x_0^-(\tau),\,p^+\right(\tau)] &=& -i \\
\left[X^i(\tau,\sigma_0),\,\mathcal{P}^{\tau j}(\tau,\sigma_1)\right] &=& i\eta^{i j}\delta(\sigma_0 - \sigma_1),
\end{eqnarray}
with all other commutators vanishing.
\subsection{String Hamiltonian}
As with the point particle, we need an operator that governs $\tau$ evolution. By the same arguments as were used before, $p^-$ govern $X^+$ evolution, and by our choice of gauge we know that $X^+ = 2\alpha'p^+\tau$. Thus, $\tau$ evolution is governed by $2\alpha'p^+\,\partial/\partial X^+$, so our Hamiltonian is given by 
\begin{equation}\label{stringphamiltonian}
\hat{H} = 2\alpha'\hat{ p}^+(\tau) \hat{p}^-(\tau).
\end{equation}
Now, we expect $p^+$ to be a constant of the motion, so we can drop the $\tau$ dependence from this expression. In addition, we can use the definition of $p^\mu$ as an integral (equation \ref{momentumconstanttau}) to write this as
\begin{equation}
\hat{H} = 2\alpha'\hat{p}^+\int_0^\pi \mathcal{P}^{\tau -}d\sigma.
\end{equation} 
We have all the tools necessary to find this in terms of the transverse coordinates and momenta. First recall our equation of motion in terms of $\mathcal{P}^{\tau \mu}$, given in equation \ref{taumomentumeom}. With our new constant $\alpha'$, this equation of motion is written as
\begin{equation}\label{taumomentumeom2}
\mathcal{P}^{\tau \mu} = \frac{1}{2\pi\alpha'}\dot{X}^\mu,
\end{equation} 
and in particular, the $-$ component is given by
\begin{equation}
\mathcal{P}^{\tau -} = \frac{1}{2\pi\alpha'}\dot{X}^-.
\end{equation} 
But in equation \ref{xdotminus} we found an explicit expression for $\dot{X}^-$. Together, this gives
\begin{eqnarray*}
\mathcal{P}^{\tau -} &=& \frac{1}{2\pi\alpha'}\frac{(\dot{\mathbf{X}}^i)^2 + (\mathbf{X}'^i)^2}{\alpha'p^+}\\
&=&\frac{\dot{X}^i\dot{X}_i + X'^iX'_i}{2\pi\alpha'^2p^+}.
\end{eqnarray*}
Using equation \ref{taumomentumeom} again, but this time with the $i$ components, we can go even further, and write
\begin{equation}
\mathcal{P}^{\tau -} = \frac{\pi}{2p^+}\mathcal{P}^{\tau i}\mathcal{P}^\tau_i + \frac{1}{8p^+\pi\alpha'^2}X'^iX'_i.
\end{equation}
Using the operator equivalent of this expression allows us to write our Hamiltonian out as
\begin{equation}\label{stringhamiltonian}
\hat{H} = \int_0^\pi\left(\alpha'\pi\hat{\mathcal{P}}^{\tau i}\hat{\mathcal{P}}^\tau_i + \frac{\hat{X}'^i\hat{X}'_i}{4\pi\alpha'}\right)d\sigma.
\end{equation}
Like with the point particle, time evolution of an operator $\hat{A}(\tau,\sigma)$ can be derived using
\begin{equation}
\frac{\partial\hat{A}(\tau,\sigma)}{\partial \tau} = -i[\hat{A}(\sigma,\tau),\hat{H}(\tau,\sigma)]
\end{equation}
Using this expression, it is clear to see that it was reasonable to assume that $p^+$ is time independent; every term in the Hamiltonian commutes with $p^+$ by our ansatz commutation relations, so $\partial \hat{p}^+ / \partial \tau = 0$. The same holds for $\hat{x}_0^-$ and for $\hat{H}$ itself. 

Next we wish to find what this Hamiltonian predicts the $\tau$ evolution will be for the $X^i$.
\begin{equation}
\frac{\partial \hat{X}^i}{\partial \tau} = - i\left[\hat{X}^i(\tau,\sigma),\,\hat{H}\right]
\end{equation}
The $\hat{X}^i$ commute with the $\hat{X}'^j$, so we only need to consider the first term of the Hamiltonian.
\begin{eqnarray}
\frac{\partial \hat{X}^i(\tau,\sigma_0)}{\partial \tau} &=& - i\left[\hat{X}^i(\tau,\sigma_0),\,\alpha'\pi\int_0^\pi\hat{\mathcal{P}}^{\tau j}(\tau,\sigma_1)\hat{\mathcal{P}}^\tau_j(\tau,\sigma_1)d\sigma_1\right]\nonumber\\
&=&-i\alpha'\pi\int_0^\pi\left(\hat{X}^i(\tau,\sigma_0)\hat{\mathcal{P}}^{\tau j}(\tau,\sigma_1)\hat{\mathcal{P}}^\tau_j(\tau,\sigma_1)-\hat{\mathcal{P}}^{\tau j}(\tau,\sigma_1)\hat{\mathcal{P}}^\tau_j(\tau,\sigma_1)\hat{X}^i(\tau,\sigma_0)\right)d\sigma_1\nonumber\\
&=&-i\alpha'\pi\int_0^\pi\bigg(\left(\hat{\mathcal{P}}^{\tau j}(\tau,\sigma_1)\hat{X}^i(\tau,\sigma_0)-\left[\hat{X}^i(\tau,\sigma_0),\,\hat{\mathcal{P}}^{\tau j}(\tau,\sigma_1)\right]\right)\hat{\mathcal{P}}^\tau_j(\tau,\sigma_1)\nonumber\\
&&\quad\quad\quad-\hat{\mathcal{P}}^{\tau j}(\tau,\sigma_1)\left(\hat{X}^i(\tau,\sigma_0)\hat{\mathcal{P}}^\tau_j(\tau, \sigma_1) + \left[\hat{X}^i(\tau,\sigma_0),\,\hat{\mathcal{P}}^\tau_j(\tau,\sigma_1)\right]\right)\bigg) d\sigma_1\nonumber\\
&=& -i\alpha'\pi\int_0^\pi\left(-i\eta^{i j}\delta(\sigma_0-\sigma_1)\hat{\mathcal{P}}^\tau_j(\tau,\sigma_1)-\hat{\mathcal{P}}^{\tau j}(\tau,\sigma_1)i\eta^i_j\delta(\sigma_0-\sigma_1)\right)d\sigma_1\nonumber\\
&=& 2\alpha'\pi\int_0^\pi\hat{\mathcal{P}}^{\tau i}(\tau,\sigma_1)\delta(\sigma_0-\sigma_1)d\sigma\nonumber\\
&=&2\alpha'\pi\hat{\mathcal{P}}^{\tau i}(\tau, \sigma_0)\label{foo8}
\end{eqnarray}
This is just the operator equivalent of the $\hat{\mathcal{P}}^{\tau \mu}$ equation of motion (\ref{taumomentumeom2}).

The derivation of the evolution of the $\mathcal{P}^{\tau i}$ follows a similar pattern, although this time it is the $\hat{\mathcal{P}}^{\tau i}$ terms of the Hamiltonian that vanish immediately, and the $\hat{X}'^i$ terms that require more care. Working through the same procedure gives the result
\begin{equation}
\frac{\partial \hat{\mathcal{P}}^{\tau i}(\tau,\sigma_0)}{\partial \tau} = \frac{1}{2\pi\alpha'}\hat{X}'^i(\tau,\sigma_0),
\end{equation}
and differentiating again allows us to reform the result
\begin{equation}
\ddot{\hat{X}}^i-\hat{X}''^i = 0.
\end{equation}
This reinforces our candidate Hamiltonian, as it has recovered the expected dynamics.

A couple more commutators that will prove useful involve the combinations or derivatives, $\hat{\dot{X}}^i \pm \hat{X}'^i$. The derivation of these commutators is relatively simple, so the results are stated without proof.
\begin{eqnarray}
\left[(\hat{\dot{X}}^i\pm\hat{X}'^i)(\tau,\sigma_0),\,(\hat{\dot{X}}^j\pm\hat{X}'^i)(\tau,\sigma_1)\right] &=& \pm4i\pi\alpha'\eta^{i j}\frac{d}{d\sigma_0}(\sigma_0-\sigma_1)\label{dpmcomm}\\
\left[(\hat{\dot{X}}^i\pm\hat{X}'^i)(\tau,\sigma_0),\,(\hat{\dot{X}}^j\mp\hat{X}'^i)(\tau,\sigma_1)\right] &=& 0\label{dmpcomm}
\end{eqnarray}
\subsection{$\hat{\alpha}_n^i$ as Operators}
We now seek a way of finding the commutators of our oscillator operators, $\hat{\alpha_n^i}$. We begin from the expression for the combinations of the $X$ derivatives given in equation \ref{sumderivatives}. In operator form this can be written
\begin{equation}
(\hat{\dot{X}}^i\pm\hat{X}'^i)(\tau,\sigma) = \sqrt{2\alpha'}\sum_{n \in \mathbb{Z}} \hat{\alpha}_n^i\cos n\sigma e^{-in(\tau\pm\sigma)}.
\end{equation}
This equation holds on the interval $[0,\,\pi]$. As such, we can construct a $2\pi$-periodic function $\hat{A}^i$ by taking the odd extension of the 'minus' form of this equation and stitching the two equations together at 0. To be precise, we define
\begin{equation}
\hat{A}^i(\tau,\sigma) := \left\{\def\arraystretch{1.2}%
\begin{array}{@{}c@{\quad}l@{}}
(\hat{\dot{X}}^i +\hat{X}'^i)(\tau,\sigma)& \sigma \in [0,\,\pi]\\
(\hat{\dot{X}}^i -\hat{X}'^i)(\tau,-\sigma)&\sigma \in [0,\,-\pi],
\end{array}
\right.
\end{equation} 
which is equivalent to the definition
\begin{equation}\label{bigAmode}
\hat{A}^i(\tau,\sigma) = \sqrt{2\alpha'}\sum_{n \in \mathbb{Z}}\hat{\alpha}_n^ie^{-in(\tau + \sigma)}
\end{equation}
Using equations \ref{dpmcomm} and \ref{dmpcomm}, we can find the commutators of this operator. When $\sigma_0$ and $\sigma_1$ are either both positive or both negative we use \ref{dpmcomm}, and when the signs differ we use \ref{dmpcomm}. This leads to the following simple expression for the commutator.
\begin{equation}\label{acomm}
\left[\hat{A}^i(\tau,\sigma_0),\,\hat{A}^j(\tau,\sigma_1)\right] = 4\pi\alpha'i\eta^{i j} \frac{d}{d\sigma}\delta(\sigma_0-\sigma_1)
\end{equation}
Using the mode expansion definition \ref{bigAmode}, we can also find an expression for the $\hat{A}$ commutators in terms of the $\hat{\alpha}_n^i$ commutators.
\begin{eqnarray*}
\left[\hat{A}^i(\tau,\sigma_0),\,\hat{A}^j(\tau,\sigma_1)\right] &=& \left[\sqrt{2\alpha'}\sum_{n \in \mathbb{Z}}\hat{\alpha}_n^ie^{-in(\tau + \sigma_0)},\sqrt{2\alpha'}\sum_{m \in \mathbb{Z}}\hat{\alpha}_m^ie^{-in(\tau + \sigma_1)}\right]\\
&=&2\alpha'\sum_{n \in \mathbb{Z}}\sum_{m\in\mathbb{Z}}e^{-i\tau(n+m)}e^{-i(n\sigma_0 + m\sigma_1)}\left[\hat{\alpha}_n^i,\,\hat{\alpha}_m^j\right]
\end{eqnarray*}
Equating this with equation \ref{acomm} gives
\begin{equation}\label{modeequation}
4\pi\alpha'i\eta^{i j} \frac{d}{d\sigma}\delta(\sigma_0-\sigma_1) = 2\alpha'\sum_{n \in \mathbb{Z}}\sum_{m\in\mathbb{Z}}e^{-i\tau(n+m)}e^{-i(n\sigma_0 + m\sigma_1)}\left[\hat{\alpha}_n^i,\,\hat{\alpha}_m^j\right].
\end{equation}
We wish to pick out particular values of $m$ and $n$, say $m'$ and $n'$. In order to do this, we observe that if we integrate an expression of the form $Ge^{ik\sigma}$ on the interval $\sigma \in [-\pi,\pi]$ for integer $k$, the result vanishes unless $k=0$ where it takes the value $2\pi G$. For all $k \ne 0$ the result includes a factor of $\sin n\pi$, and so vanishes.  In this case, we can use two such integration operations, one each for $\sigma_0$ and $\sigma_1$, to pick out our particular values of $m$ and $n$. Note that this was the reason for defining our $\hat{A}^i$, as we required an interval of length $2\pi$ to use this result.

 Applying the operation
\begin{equation}
\frac{1}{2\pi}\int_{-\pi}^\pi d\sigma_0e^{in'\sigma_0}\frac{1}{2\pi}\int_{-\pi}^\pi d\sigma_1 e^{im'\sigma_1}
\end{equation}
to both sides of \ref{modeequation} gives 
\begin{equation}
\frac{i\alpha'}{\pi}\eta^{i j}\int_{-\pi}^{\pi}d\sigma_0 e^{-in'\sigma_0}\frac{d}{d\sigma_0}\int_{-\pi}^\pi d\sigma_1 e^{im'\sigma_1}\delta(\sigma_0-\sigma_1) = 2\alpha'e^{-i\tau(n+m)}\left[\hat{\alpha}_{n'}^i,\hat{\alpha}_{m'}^j\right].
\end{equation}
To deal with the left hand side of this equation, we first carry out the $\sigma_1$ integration to remove the delta function, leaving
\begin{equation}
\frac{i\alpha'}{\pi}\eta^{ij}\int_{-\pi}^\pi d\sigma_0\, e^{in\sigma_0}\frac{d}{d\sigma_0}e^{im\sigma_0},
\end{equation}
and then we can carry out the $\sigma_0$ differentiation and integration to give
\begin{equation}
-2m'\alpha'\eta^{ij}\delta(m'+n') = 2\alpha'e^{-i\tau(m+n)}\left[\hat{\alpha}_{n'}^i,\hat{\alpha}_{m'}^j\right].
\end{equation}
Since $m'$ and $n'$ only take integer values, we can use a Kronecker instead of a Dirac delta. We also choose to revert back to the symbols $m$ and $n$ for simplicity. Thus, making the commutator the subject of this expression we now have
\begin{equation}\label{alphacommutation}
\left[\hat{\alpha_n}^i,\hat{\alpha}_m^j\right] = n\eta^{ij}\delta_{n+m,0}.
\end{equation}
It can sometimes be useful to instead write this as 
\begin{equation}
\left[\hat{\alpha_n}^i,\hat{\alpha}_{-m}^j\right] = n\eta^{ij}\delta_{n,m}.
\end{equation}

As we did when we concluded section \ref{lceom}, we can now give a full set of equations for $X^\mu$ in light-cone coordinates, but this time in operator form.
\begin{eqnarray}
\hat{X}^+(\tau,\sigma) &=& 2\alpha'\hat{p}^+\tau\label{operatorexpansionb}\\
\hat{X}^-(\tau,\sigma) &=& \hat{x}_0^- + \sqrt{2\alpha'}\hat{\alpha}_0^i\tau + i\sqrt{2\alpha}\sum_{n=0}^\infty\frac{1}{n}\hat{\alpha}_n^-e^{-in\tau}\cos n\sigma\\
\hat{X}^i(\tau,\sigma) &=& \hat{x}_0^i +\sqrt{2\alpha'}\hat{\alpha}_0^i\tau + i\sqrt{2\alpha'}\sum_{n=0}^\infty\frac{1}{n}\hat{\alpha}_n^ie^{-in\tau}\cos n\sigma\\
\text{where}\quad\hat{\alpha}_n^- &=& \frac{1}{2p^+\sqrt{2\alpha'}}\sum_{a \in \mathbb{Z}}\sum_{i=2}^{d-1}\hat{\alpha}^i_{n-a}\alpha^i_a\label{operatorexpansione}
\end{eqnarray}
\subsection{Annihilation and Creation Operators}\label{annihilation}
When we first introduced the set of constants $\alpha_n^\mu$ back in section \ref{wegs} it was in order to reduce the pair of sums over the natural numbers to a single sum over the integers. Now however, it is instructive to reverse this process. Previously, as the $a_n^\mu$ we began with were complex constants, the 'conjugation' process was simply complex conjugation. Now, however, we are working with operators, and the equivalent process is Hermitian conjugation. With this slight difference, our change of variables \ref{atoalpha1} and \ref{atoalpha2} becomes
\begin{eqnarray}
\hat{\alpha}_n^i &=& \sqrt{n}\hat{a}_n^i\label{atoalpha3}\\
\hat{\alpha}_{-n}^i &=& \sqrt{n}(\hat{a}_{-n}^i)^\dagger\label{atoalpha4}.
\end{eqnarray} 
We now have two sets of operators, $\hat{a}_n^i$ and $(\hat{a}_n^i)^\dagger$ for $n>1$. This choice of notation should be familiar from the study of simple harmonic oscillators in quantum mechanics\cite{gasiorowicz}. There, the symbols $\hat{a}$ and $\hat{a}^\dagger$ are commonly used to denote annihilation and creation operators respectively. We will now demonstrate that our $\hat{a}_n^i$ and $(\hat{a}_n^i)^\dagger$ are in fact an infinite set of annihilation and creation operators.


For the simple harmonic oscillator the characteristic commutation relations between the annihilation and creation operators are
\begin{equation}
[\hat{a},\,\hat{a}^\dagger] = 1,\quad[\hat{a},\,\hat{a}]=[\hat{a}^\dagger,\,\hat{a}^\dagger]=0.
\end{equation}
(recall that we have set $\hbar$ equal to $1$.) It is easy to see that we will obtain the equivalent commutation relations for $[\hat{a}_n^i,\,\hat{a}_m^j]$ and $[(\hat{a}_n^i)^\dagger,\,(\hat{a}_m^j)^\dagger]$, as this is a case of equation \ref{alphacommutation} where either $m$ and $n$ are both positive, or both negative. This precludes the possibility of $m-n=0$, so the $\delta$ function on the right hand side causes the commutator to vanish. For the commutator of $\hat{\alpha}_n^i$ and $(\hat{\alpha}_m^j)^\dagger$, we can use our definitions \ref{atoalpha3} and \ref{atoalpha4} explicitly in our $\hat{\alpha}$ commutation relation \ref{alphacommutation} to get
\begin{equation}
\left[\sqrt{n}\hat{a}_n^i,\,\sqrt{m}(\hat{a}_m^j)^\dagger\right] = m\eta^{ij}\delta_{n,m}.
\end{equation}
Then, since the Kronecker $\delta$ will force the commutator to vanish unless $m=n$, this simplifies to
\begin{equation}
[\hat{a}_n^i,\,\hat{a}_m^j] = \eta^{i j}\delta_{n,m}.
\end{equation} 
This is clearly the most natural extension of the commutation relation for the simple harmonic oscillator. We would not expect the operators corresponding to different values of $i$ to interfere, nor would we expect operators corresponding to different modes $n$ to interfere. Since the $i$ are spacetime coordinates, the Minkowski metric is the natural way of separating them out, and since the $n$ are just natural numbers the Kronecker delta is used. 

Since the sets of $\hat{a}_n^i$, $(\hat{a}_n^i)^\dagger$ satisfy the correct commutation relations, we are safe to consider them as annihilation and creation operators. We can also now find the mode expansion of the transverse coordinate operators $X^i$ in terms of annihilation and creation operators.
\begin{equation}
\hat{X}^i(\tau, \sigma) = \hat{x}_0^i + 2\alpha'\hat{p}^i\tau + i\sqrt{2\alpha'}\sum_{n=1}^\infty\frac{1}{n}\left(\hat{a}_n^ie^{-in\tau}-(\hat{a}_n^i)^\dagger e^{in\tau}\right)\cos n\sigma,
\end{equation}
where we have recalled that in section \ref{wegs} we defined
\begin{equation}
\alpha_0^\mu = \sqrt{2\alpha'}p^\mu.
\end{equation}
Since we are not using the set of $\hat{\alpha}_n^\mu$ operators, it is convenient to return the more explicit notation. 

When we started our treatment of the quantum string, we considered $\hat{X}^i$ to be one of our fundamental operators. Now we see that these operators can be replaced by our infinite sets of annihilation and creation operators $\hat{a}_n^i$, $(\hat{a}_n^i)^\dagger$, and centre-of-mass operators ($\hat{x}_0^i$, $\hat{p}^i$). In fact, since (by equation \ref{foo8}) $\hat{\mathcal{P}}^{\tau i}$ is proportional to the $\tau$ derivative of $\hat{X}^i$, we can write
\begin{equation}
\mathcal{P}^{\tau i}(\tau,\sigma) = \frac{1}{\pi}\hat{p}^i +\frac{1}{\pi\sqrt{2\alpha'}}\sum_{n=1}^\infty\left(\hat{a}_n^ie^{-in\tau}+(\hat{a}_n^i)^\dagger e^{-in\tau}\right)\cos{n\sigma},
\end{equation}
which clearly shows that we can remove $\hat{\mathcal{P}}^{\tau i}$ from our list of fundamental operators. Thus, our full (and complete) set of fundamental operators now consists of
\begin{equation}
\hat{x}_0^-,\quad \hat{p}^+,\quad \hat{x}_0^i,\quad \hat{p}^i, \quad \hat{a}_n^i, \quad (\hat{a}_n^i)^\dagger,
\end{equation}
for all $n \in \mathbb{N}$, for all $2\leq i\leq d$. 

The first four of these are a description of the centre-of-mass motion of the string, and the final two are an infinite set that describe the oscillation of the string.

Moving into the next section we will revert to the $\alpha$ operators. The substance of this section can be summarised as 
\begin{itemize}
\item{$\hat{\alpha}_n^i$ act as \emph{annihilation} operators for $n>0$}
\item{$\hat{\alpha}_n^i$ act as \emph{creation} operators for $n<0$}
\end{itemize}
\subsection{Transverse Virasoro Operators, and the Virasoro Algebra}\label{secvirasoro}
Returning to our set of light-cone coordinate mode expansions \ref{operatorexpansionb}-\ref{operatorexpansione}, we can see that the expressions for for $\hat{X}^+$ and $\hat{X}^i$ operators are relatively simple, while the expression for the $\hat{X}^-$ operator involves an infinite set of operators $\hat{\alpha}_n^-$ that \emph{each} involve an infinite sum of $\hat{\alpha}_p^i$. Fortunately, these operators actually behave in a fairly simple and interesting way. We use our equation \ref{operatorexpansione} to define the \emph{transverse Virasoro operator} $L_n^\perp$ by
\begin{equation}\label{virasorosum}
\hat{L}_n^\perp = \frac{1}{2}\sum_{i=2}^{d-1}\sum_{a\in\mathbb{Z}}^{d-1}\hat{\alpha}_{n-a}^i\hat{\alpha}_a,
\end{equation}
which allows us to write
\begin{equation}
\sqrt{2\alpha'}\hat{\alpha}_n^- = \frac{1}{\hat{p}^+}\hat{L}_n^\perp.
\end{equation}
Because of the particular choice of constants here, we can write the mode expansion of $\hat{X}^-$ as
\begin{equation}
\hat{X}^- = \hat{x}_0^-+\frac{1}{p^+}\hat{L}_0^\perp + \frac{i}{p^+}\sum_{n \ne 0}\hat{L}_n^\perp e^{-in\tau}\cos n\sigma.
\end{equation}
Initially we will only discuss $\hat{L}_n^\perp$ for $n \ne 0$, as that turns out to be have issues that will require special treatment. Firstly, since $n \ne 0$ so we know that any pair $\hat{\alpha}_{n-a}^i,\,\hat{\alpha}_a^i$ will commute
\begin{eqnarray*}
(\hat{L}_n^\perp)^\dagger &=& \left(\frac{1}{2}\sum_{i=2}^{d-1}\sum_{a \in \mathbb{Z}}\hat{\alpha}_{n-a}^i\hat{\alpha}_{a}^i\right)^\dagger\\
&=&\frac{1}{2}\sum_{i=2}^{d-1}\sum_{a \in \mathbb{Z}}(\hat{\alpha}_a^i)^\dagger(\hat{\alpha}_{n-a}^i)^\dagger\\
&=&\frac{1}{2}\sum_{i=2}^{d-1}\sum_{a \in \mathbb{Z}}(\hat{\alpha}_{-a}^i)(\hat{\alpha}_{a-n}^i)^\dagger\\
&=&\frac{1}{2}\sum_{i=2}^{d-1}\sum_{a \in \mathbb{Z}}(\hat{\alpha}_{a-n}^i)(\hat{\alpha}_{-a}^i)^\dagger\\
&=&\frac{1}{2}\sum_{i=2}^{d-1}\sum_{a \in \mathbb{Z}}(\hat{\alpha}_{-a-n}^i)(\hat{\alpha}_{a}^i)^\dagger\\
&=&\hat{L}_{-n}^\perp
\end{eqnarray*}
The next natural step is to try to find commutation relations for the Virasoro operators. Trying to find the commutator between two Virasoro operators by brute force is possible, but results in extremely lengthy computations. It is much easier to use the intermediate step of finding the commutator between an $\hat{L}_n^\perp$ and an $\alpha_n^i$ first. Once more using the commutator identity $[\hat{A}\hat{B},\hat{C}]= [\hat{A},\hat{C}]\hat{B} + \hat{A}[\hat{B},\hat{C}]$, we have  
\begin{eqnarray*}
[\hat{L}_n^\perp,\,\hat{\alpha}_m^j] &=& \left[\left(\frac{1}{2}\sum_{i=2}^{d-1}\sum_{a\in\mathbb{Z}}\hat{\alpha}_{n-a}^i\hat{\alpha}_a^i\right),\,\hat{\alpha}_m^j\right]\\
&=& \frac{1}{2}\sum_{i=2}^{d-1}\sum_{a\in\mathbb{Z}}\left[\hat{\alpha}_{n-a}^i\hat{\alpha}_a^i,\hat{\alpha}_m^j\right]\\
&=& \frac{1}{2}\sum_{i=2}^{d-1}\sum_{a\in\mathbb{Z}}\Big(\hat{\alpha}_{n-a}^i\left[\hat{\alpha}_a^i,\hat{\alpha}_m^j\right] + \left[\hat{\alpha}_{n-a}^i,\hat{\alpha}_m^j\right]\hat{\alpha}_a^i\Big) \\
&=&\frac{1}{2}\sum_{i=2}^{d-1}\sum_{a\in\mathbb{Z}}\Big(\hat{\alpha}_{n-a}^i a \eta^{i j}\delta_{n+a,0} + (n-a)\eta^{ij}\delta_{n-a+m,0}\hat{\alpha}_a^i\Big)\\
&=&\frac{1}{2}\sum_{a\in\mathbb{Z}}\Big(a\delta_{m+a,0}\hat{\alpha}_{n-a}^j + (n-a)\delta_{n-a+m,0}\hat{\alpha}_a^j\Big)\\
&=&\frac{1}{2}(-m\hat{\alpha}_{n+m}^j-m\hat{\alpha}_{n+m}^j)\\
&=&-m\hat{\alpha}_{n+m}^j. 
\end{eqnarray*}
We can now use this result to calculate the commutator of two Virasoro operators.
\begin{eqnarray*}
[\hat{L}_n^\perp,\,\hat{L}_m^\perp] &=& \left[\left(\frac{1}{2}\sum_{i=2}^{d-1}\sum_{a\in\mathbb{Z}}\hat{\alpha}_{n-a}^i\hat{\alpha}_a^i\right),\,\hat{L}_m^\perp\right]\\
&=& \frac{1}{2}\sum_{i=2}^{d-1}\sum_{a\in\mathbb{Z}}\left[\hat{\alpha}_{n-a}^i\hat{\alpha}_a^i,\hat{L}_m^\perp\right]\\
&=&\frac{1}{2}\sum_{i=2}^{d-1}\sum_{a\in\mathbb{Z}}\left(\hat{\alpha}_{n-a}^i\left[\hat{\alpha}_a^i,\,\hat{L}_m^\perp\right] + \left[\hat{\alpha}_{n-a}^i,\,\hat{L}_m^\perp\right]\hat{\alpha}_a^i\right)\\
&=&\frac{1}{2}\sum_{i=2}^{d-1}\sum_{a\in\mathbb{Z}}\left(a\hat{\alpha}_{n-a}^i\hat{\alpha}_{a+m}^i + (n-a)\hat{\alpha}_{n-a+m}^i\hat{\alpha}_a^i\right)
\end{eqnarray*}
From here we must proceed with a little more caution, as the results differ depending on whether the pairs of $\alpha$ operators commute. We first split each of the two terms into a pair of sums, one for $a \geq 0$ and one for $a < 0$, as follows
\begin{eqnarray}\label{foursums}
&=& \frac{1}{2}\sum_{i=2}^{d-1}\Big(\sum_{a \geq0}a\hat{\alpha}_{n-a}^i\hat{\alpha}_{a+m}^i + \sum_{a < 0}a\hat{\alpha}_{a+m}^i\hat{\alpha}_{n-a}^i \\
&& \quad + \sum_{a \geq 0}(n-a)\hat{\alpha}_{n-a+m}\hat{\alpha}_a^i + \sum_{a < 0}(n-a)\hat{\alpha}_a^i\hat{\alpha}_{n-a+m}^i \Big).\nonumber
\end{eqnarray}
Now, if $m+n \ne0$, all pairs of $\alpha$ commute, so by some simple summation shifting we find 
\begin{eqnarray}\label{mneqnvirasorocomm}
\left[\hat{L}_n^\perp,\,\hat{L}_m^\perp\right] &=& (n-m)\frac{1}{2}\sum_{i=2}^{d-1}\sum_{a\in\mathbb{Z}}\hat{\alpha}_{n-a+m}^i\hat{\alpha}_a^i\\
&=& (n-m)\hat{L}_{n+m}^\perp.
\end{eqnarray}
On the other hand, if $n+m=0$, this gives us a constraint equation which we can use to reduce equation \ref{foursums}.
\begin{eqnarray}\label{foursums2}
\left[\hat{L}_n^\perp,\,\hat{L}_{-n}^\perp\right]&=& \frac{1}{2}\sum_{i=2}^{d-1}\Big(\sum_{a \geq0}a\hat{\alpha}_{n-a}^i\hat{\alpha}_{a-n}^i + \sum_{a < 0}a\hat{\alpha}_{a-n}^i\hat{\alpha}_{n-a}^i \\
&& \quad + \sum_{a \geq 0}(n-a)\hat{\alpha}_{-a}\hat{\alpha}_a^i + \sum_{a < 0}(n-a)\hat{\alpha}_a^i\hat{\alpha}_{-a}^i \Big).\nonumber
\end{eqnarray}
These sums look completely incomparable, but we can change our summation variable for each of them to create some consistency. We choose to force each rightmost $\hat{\alpha}$ operator to be $\hat{\alpha}_a^i$. In the four sums we shift
\begin{itemize}
\item{1) $a \to n+a$}
\item{2) $a \to n-a$}
\item{3) $a \to a$}
\item{4) $a \to -a$}
\end{itemize} 
This results in
\begin{eqnarray}\label{foursums3}
\left[\hat{L}_n^\perp,\,\hat{L}_{-n}^\perp\right]&=& \frac{1}{2}\sum_{i=2}^{d-1}\Big(\sum_{a=-n}^\infty (n+a)\hat{\alpha}_{-a}^i\hat{\alpha}_{a}^i + \sum_{a = n+1}^\infty (n-a)\hat{\alpha}_{-a}^i\hat{\alpha}_{a}^i \\
&& \quad + \sum_{a = 0}^\infty(n-a)\hat{\alpha}_{-a}\hat{\alpha}_a^i + \sum_{a =1}^\infty(n+a)\hat{\alpha}_{-a}^i\hat{\alpha}_a^i \Big).\nonumber
\end{eqnarray}
If we assume $n>0$, the first sum is the only one where the summation index takes negative values. By splitting it into two sums, one for $-n< a<0$, and one for $a>0$, and sending $a\to -a$ in the first one, we can make all summation indices positive. With these changes, the first sum becomes
\begin{equation}
\frac{1}{2}\sum_{i=2}^{d-1}\sum_{a=-n}^\infty (n+a)\hat{\alpha}_{-a}^i\hat{\alpha}_{a}^i = \frac{1}{2}\sum_{i=2}^{d-1}\left(\sum_{a=0}^n(n-a)\hat{\alpha}_a^i\hat{\alpha}_{-a}^i + \sum_{a=1}^\infty(n+a)\hat{\alpha}_{-a}^i\hat{\alpha}_a^i\right).
\end{equation}
Then, in order to get all of the rightmost $\hat{\alpha}$ to be $\hat{\alpha}_n^i$ again, we use the commutator in the first sum, giving
\begin{eqnarray}\label{threesums}
&&\frac{1}{2}\sum_{i=2}^{d-1}\left(\sum_{a=0}^n(n-a)\left[\hat{\alpha}_a^i,\hat{\alpha}_{-a}^i\right] +\sum_{a=0}^n(n-a)\hat{\alpha}_a^i\hat{\alpha}_{-a}^i + \sum_{a=1}^\infty(n+a)\hat{\alpha}_{-a}^i\hat{\alpha}_a^i\right)\nonumber\\
&=&\frac{1}{2}\sum_{i=2}^{d-1}\left(\sum_{a=0}^n(n-a)(-n)\eta^{i i} +\sum_{a=0}^n(n-a)\hat{\alpha}_a^i\hat{\alpha}_{-a}^i + \sum_{a=1}^\infty(n+a)\hat{\alpha}_{-a}^i\hat{\alpha}_a^i\right)\nonumber\\
&=&\frac{1}{2}(d-2)\sum_{a=0}^n n(a-n)+\frac{1}{2}\sum_{i=2}^{d-1}\left(\sum_{a=0}^n(n-a)\hat{\alpha}_a^i\hat{\alpha}_{-a}^i + \sum_{a=1}^\infty(n+a)\hat{\alpha}_{-a}^i\hat{\alpha}_a^i\right).
\end{eqnarray}
The first sum here is a constant, and is identically equal to $(a^3-a)/12$. When this expression is substituted back into \ref{foursums3}, the second term from equation \ref{threesums} combines with the second term from equation \ref{foursums3} to give a single sum from 0 to $\infty$. The third term from \ref{threesums} and the fourth term from \ref{foursums3} are identical, so they add. Thus we are left with
\begin{eqnarray}
\left[\hat{L}_n^\perp,\,\hat{L}_{-n}^\perp\right]&=& \sum_{i=2}^{d-1}\Big(\sum_{a=0}^\infty (n-a)\hat{\alpha}_{-a}^i\hat{\alpha}_{a}^i +\sum_{a = 1}^\infty (n+a)\hat{\alpha}_{-a}^i\hat{\alpha}_{a}^i\Big) + \frac{d-2}{24}(a^3-a).
\end{eqnarray}
We can separate out the $a=0$ term from the first sum, and then recombine the remaining terms. By doing this we see that the $a$ terms cancel out, leaving
\begin{equation}\label{virasorocomm1}
\left[\hat{L}_n^\perp,\,\hat{L}_{-n}^\perp\right]= \sum_{i=2}^{d-1}\Big(n\hat{\alpha}_0^i\hat{\alpha}_0^i + \sum_{a=1}^\infty 2n\hat{\alpha}_{-a}^i\hat{\alpha}_{a}^i\Big) + \frac{d-2}{24}(a^3-a).
\end{equation}
We now return to the case for $n=0$. $\hat{L}_0^\perp$ is actually our string Hamiltonian, as in equation \ref{stringphamiltonian}. Using the definition of $\alpha_0^\mu$ in equation \ref{pmudefn}
\begin{eqnarray*}
\hat{L}_0^\perp &=& p^+\sqrt{2\alpha'}\hat{\alpha}_0^-\\
&=& 2\alpha'\hat{p}^+\hat{p}^-\\ &=& \hat{H}. \end{eqnarray*}
However, if we were to try to define a space of states, this definition of this operator would immediately fail. If we expand the sum \ref{virasorosum} for the $n=0$ case slightly
\begin{equation}
\hat{L}_0^\perp = \frac{1}{2}\sum_{i=2}^{d-1}\left(\hat{\alpha}_0^i\hat{\alpha}_0^i + \sum_{a=1}^\infty\hat{\alpha}_{-a}^i\hat{\alpha}_a^i + \sum_{a=1}^\infty\hat{\alpha}_a^i\hat{\alpha}_{-a}^i\right),
\end{equation} 
we see that the quadratic combinations of $\hat{\alpha}$ from the second sum inside the bracket have creation operators as the right-most operator. Thus, if this operator was applied to some vacuum state $\ket{0}$ (which any state space must include), the creation operator would act directly on the vacuum state, and the result would not be well defined\cite{peskin}. Note that this problem only arises for $n=0$. To see this consider the definition of $\hat{L}_n^\perp$ as given in equation \ref{virasorosum}. For all $n\ne0$ each term of the sum will commute, so we can freely reorder the ones that would have a creation operator on the right.
 
The process used for dealing with this kind of issue (in general quantum field theory as well as string theory) is called normal ordering. The order of the operators is switched by using the commutator \ref{alphacommutation},
\begin{equation}
\frac{1}{2}\sum_{i-2}^{d-1}\hat{\alpha}_a^i\hat{\alpha}_{-a}^i = \sum_{i-2}^{d-1}\sum_{a=1}^\infty\left(\hat{\alpha}_{-a}^i\hat{\alpha}_a^i + \left[\alpha_a^i,\,\alpha_{-a}^i\right]\right).
\end{equation} 
Since in the commutator the lower indices are fixed to be equal and opposite-signed, the Kronecker $\delta$ is automatically satisfied, and the commutator term becomes
\begin{equation}
\frac{1}{2}\sum_{i=2}^{d-1}\sum_{a=1}^\infty a\eta^{ii}.
\end{equation}
We can now carry out the the sum over $i$, leaving
\begin{equation}
\frac{d-2}{2}\sum_{a=1}^\infty a.
\end{equation} 
With this the expansion of $\hat{L}_0^\perp$ is given by
\begin{equation}
\hat{L}_0^\perp = \sum_{i=2}^{d-1}\left(\frac{\hat{\alpha}_0^i\hat{\alpha}_0^i}{2} + \sum_{a=1}^\infty\hat{\alpha}_{-a}^i\hat{\alpha}_a^i\right) + \frac{d-2}{2}\sum_{a=1}^\infty a
\end{equation}
This is clearly an issue. The sum of the natural numbers is divergent, and therefore not defined. We could potentially deal with the issue by just removing this term. While it is divergent, and tending to infinity, it is also a constant term. Two operators that differ only by a constant term are, in many cases, functionally identical. While compelling in theory, this idea doesn't work. In order to understand why it fails we consider the mass-squared operator for the string states.

Classically, the relativistic mass-squared, $M^2$, is simply minus the square of the momentum Lorentz vector, $p$. Similarly, the operator form in light-cone coordinates is given by
\begin{equation}
\hat{M}^2 = 2\hat{p}^+\hat{p}^- - \hat{p}^i\hat{p}_i.
\end{equation}
Notice that the first term is proportional to the zeroth Virasoro operator, $L_0^\perp$. Substituting this in, and recalling that $\hat{\alpha}_0^i = \sqrt{2\alpha'}p^i$, we obtain
\begin{eqnarray}
\hat{M}^2 &=& \frac{1}{\alpha'}\left(\sum_{i=2}^{d-1}\left(\frac{\hat{\alpha}_0^i\hat{\alpha}_0^i}{2} + \sum_{a=1}^\infty\hat{\alpha}_{-a}^i\hat{\alpha}_a^i\right) + \frac{d-2}{2}\sum_{a=1}^\infty a\right) - \hat{p}^i\hat{p_i}\\
&=&\frac{1}{\alpha'}\sum_{i=2}^{d-1}\left(\alpha'\hat{p}^i\hat{p}^i + \sum_{a=1}^\infty\hat{\alpha}_{-n}^i\hat{\alpha}_n^i-\alpha'\hat{p}^i\hat{p}^i\right) + \frac{d-2}{2\alpha}\sum_{a=1}^\infty a\\
&=& \frac{1}{\alpha'}\left(\sum_{i=2}^{d-1}\sum_{a=1}^\infty\hat{\alpha}_{-n}^i\hat{\alpha}_n^i + \frac{d-2}{2}\sum_{a=1}^\infty a\right).
\end{eqnarray}
So the mass-squared is shifted by our divergent constant. 

While we deal with this issue we will subtly redefine some of our notation. In order to isolate the divergent term, we choose to define 
\begin{equation}
\hat{L}_0^\perp = \sum_{i=2}^{d-1}\left(\frac{\hat{\alpha}_0^i\hat{\alpha}_0^i}{2} + \sum_{a=1}^\infty\hat{\alpha}_{-a}^i\hat{\alpha}_a^i\right).
\end{equation}
With this as our definition for the zeroth Virasoro operator, we lose our direct connection between $\hat{p}^-$ and $\hat{L}_n^\perp$. By labelling our divergent term $\lambda$, we now have
\begin{equation}\label{lambdapminus}
2\alpha\hat{p}^- = \frac{\hat{L}_0^\perp + \lambda}{p^+}.
\end{equation}
A useful side effect of this change of notation is a simpler expression for the Virasoro operator commutator. By noting that the sum over $i$ in equation \ref{virasorocomm1} is equal to our new definition of $\hat{L}_0^\perp$, we see that it is the same as the case for $n + m \ne 0$, given in \ref{mneqnvirasorocomm}. The constant term then arises as a separate term that appears only when $m=-n$ We can now write
\begin{equation}
\left[\hat{L}_n^\perp,\,\hat{L}_m^\perp\right] = (n-m)\hat{L}_{n+m}^\perp + \frac{d-2}{12}(m^3-m)\delta_{m+n,0}.
\end{equation}
This is the defining relation for the Virasoro algebra, which is an example of a Lie algebra\cite{frenkel}. Within the realms of string theory, Virasoro operators generate reparameterizations of the world sheet.
\subsection{The Dimensionality of Spacetime in Bosonic String Theory}\label{dimension}
To resolve our divergent term, we will use the conserved angular-momentum charges found in section \ref{lorentzcharges}. In operator form, these charges are given by
\begin{equation}
\hat{M}^{\mu \nu} = \int_C\left(\hat{X}_\mu\hat{\mathcal{P}}_\nu^\tau - \hat{X}_\nu\mathcal{P}_\mu^\tau\right)d\sigma + \int_C\left(\hat{X}_\mu\hat{\mathcal{P}}_\nu^\sigma - \hat{X}_\nu\hat{\mathcal{P}}_\mu^\sigma\right)d\tau,
\end{equation}
for some curve C which begins on the $\sigma =0$ boundary of the worldsheet and ends on the $\sigma=\pi$ boundary. We can choose a curve of constant $\tau$, and recall the relation between $\hat{\mathcal{P}}_\mu^\tau$ and $\hat{\dot{X}}_\mu$, to instead write
\begin{equation}
\hat{M}^{\mu\nu} =\frac{1}{2\pi\alpha'}\int_0^\pi\left(\hat{X}^\mu\hat{\dot{X}}^\nu - \hat{X}^\nu\hat{\dot{X}}^\mu\right)d\sigma
\end{equation}

Since these charges are angular momenta, they must follow the general angular momentum Lie algebra (the Lorentz algebra) which is characterised by operators $\hat{A}^{\mu\nu}$ with the commutation relation
\begin{equation}\label{lalgebra}
[\hat{A}^{\alpha\beta},\,\hat{A}^{\gamma\delta}] = i\eta^{\alpha\gamma}\hat{A}^{\beta\delta}-i\eta^{\beta\gamma}\hat{A}^{\alpha\delta} + i\eta^{\alpha\delta}\hat{A}^{\gamma\beta}-i\eta^{\beta\delta}\hat{A}^{\gamma\alpha}.
\end{equation}
If the operators $\hat{M}^{\mu\nu}$ were to fail to satisfy this relation, this string theory would fail to be Lorentz invariant. It is this fact that will allow us to find the required dimensionality of spacetime for bosonic string theory. The particular relation that we will use is the commutator $[\hat{M}^{i-},\,\hat{M}^{j-}]$. In this case relation \ref{lalgebra} becomes
\begin{equation}
[\hat{M}^{i-},\,\hat{M}^{j-}] = i\eta^{ij}\hat{M}^{--}-i\eta^{-j}\hat{M}^{i-} + i\eta^{i-}\hat{M}^{j-}-i\eta^{--}\hat{M}^{ji}.
\end{equation}
All $\eta^{ij}$ vanish immediately except for $\eta^{ij}$, and for $i\ne j$ we have
\begin{equation}
[\hat{M}^{i-},\,\hat{M}^{j-}] = 0.
\end{equation}
The goal is to find an expression for $[\hat{M}^{i-},\,\hat{M}^{j-}]$ in operator form, and then the vanishing condition will force us to fix a value for $d$. First we must find an expression for $\hat{M}^{i-}$.

The mode expansions of $\hat{X}^\mu$ and $\hat{\dot{X}}^\mu$ are
\begin{eqnarray}
\hat{X}^\mu(\tau,\sigma) &=& \hat{x}_0^\mu + \sqrt{2\alpha'}\hat{\alpha}_0^\mu\tau+i\sqrt{2\alpha'}\sum_{n \ne 0}\frac{1}{n}\hat{\alpha}_n^\mu e^{-in\tau}\cos n\sigma\\
\hat{\dot{X}}^\mu(\tau,\sigma) &=& \sqrt{2\alpha'}\sum_{n =-\infty}^\infty\hat{\alpha}_n^\mu e^{-in\tau} \cos n\sigma
\end{eqnarray}
It looks like an expanded expression for $M^{\mu\nu}$ would be very difficult to work with, but fortunately we can ignore many of these terms. Since we know that the overall operator is $\tau$ independent, any terms that have $\tau$ dependence must sum to zero. The only terms in the product $\hat{\dot{X}}^\mu\hat{X}^\nu$ that have no $\tau$ dependence are 
\begin{equation}
\sqrt{2\alpha'}\hat{x}_0^\mu\hat{\alpha}_0^\nu + i2\alpha'\sum_{n \ne 0}\frac{1}{n}\hat{\alpha}_{-n}^\mu\hat{\alpha}_{n}^\nu\cos^2n\sigma. 
\end{equation}
Thus, our Lorentz charges become 
\begin{equation}
\hat{M}^{\mu\nu} = \frac{1}{\pi}\int_0^\pi\left(\frac{1}{\sqrt{2\alpha'}}\left(\hat{x}_0^\mu\hat{\alpha}_0^\nu-\hat{x}_0^\nu\hat{\alpha}_0^\mu\right) -i\sum_{n=1}^\infty\frac{1}{n}\left(\hat{\alpha}_{-n}^\mu\hat{\alpha}_n^\nu-\hat{\alpha}_{-n}^\nu\hat{\alpha}_{n}^\mu\right)\cos^2n\sigma\right)d\sigma.
\end{equation}

Each of the first two terms terms gain a factor of $\pi$ on integrating between 0 and $\pi$, which then cancels with the $1/\pi$. We also have that $\hat{\alpha}_0^\mu = \sqrt{2\alpha'}\hat{p}^\mu$ so the first pair of terms become
\begin{equation}
\hat{x}_0^\mu\hat{p}^\nu-\hat{x}^\nu\hat{p}^\mu.
\end{equation}
For the sum, we have
\begin{eqnarray*}
\int_0^\pi\sum_{n\ne0}\frac{1}{n}\left(\hat{\alpha}_{-n}^\mu\hat{\alpha}_{n}^\nu-\hat{\alpha}_{-n}^\nu\hat{\alpha}_{n}^\mu\right)\cos^2n\sigma d\sigma &=&\sum_{n\ne0}\frac{1}{n}\left(\hat{\alpha}_{-n}^\mu\hat{\alpha}_n^\mu-\hat{\alpha}_{-n}^\nu\hat{\alpha}_{n}^\mu\right)\left(\frac{\pi}{2} + \frac{\sin 2\pi n}{4 n}\right)\\
&=& \sum_{n=1}^\infty\left(\frac{1}{n}\left(\hat{\alpha}_{-n}^\mu\hat{\alpha}_{n}^\nu-\hat{\alpha}_{-n}^\nu\hat{\alpha}_{n}^\mu\right)\frac{\pi}{2} - \frac{1}{n}\left(\hat{\alpha}_{n}^\mu\hat{\alpha}_{-n}^\nu-\hat{\alpha}_{n}^\nu\hat{\alpha}_{-n}^\mu\right)\frac{\pi}{2} \right)
\end{eqnarray*} 
If we were still discussing the classical case, the $\alpha$ modes would have commuted, so this would have simplified nicely to
\begin{equation}
M^{\mu\nu} = x_0^\mu p^\nu-x_0^\nu p^\mu -i\sum_{n=1}^\infty\frac{1}{n}\left(\alpha_{-n}^\mu\alpha_n^\nu-\alpha_{-n}^\nu\alpha_{n}^\mu\right).
\end{equation}
Since we are working with quantum operators however, they do not commute. We can however use this to construct an ansatz for $\hat{M}^{i-}$, and attempt to modify it to meet our conditions. As we process we will denote our ansatz by $\hat{\Lambda}^{i-}$.
\begin{equation}
\hat{\Lambda}^{i-} = \hat{x}_0^i \hat{p}^--\hat{x}_0^- \hat{p}^i +i\sum_{n=1}^\infty\frac{1}{n}\left(\hat{\alpha}_{-n}^i\hat{\alpha}_n^--\hat{\alpha}_{-n}^-\hat{\alpha}_{n}^i\right).
\end{equation}
We can use the Virasoro operators to write this as
\begin{equation}
\hat{\Lambda}^{i-} = \hat{x}_0^i \hat{p}^--\hat{x}_0^- \hat{p}^i +\frac{i}{\sqrt{2\alpha'}\hat{p}^+}\sum_{n=1}^\infty\frac{1}{n}\left(\hat{\alpha}_{-n}^i\hat{L}_n^\perp-\hat{L}_{-n}^\perp\hat{\alpha}_{n}^i\right).
\end{equation}
Firstly, we require $\hat{\Lambda}^{i-}$ to be Hermitian. As it is set up this is not a problem for the second term since $[\hat{x}_0^-,\,\hat{p}^i]=0$. Similarly, since $(\hat{\alpha}_n^i)^\dagger = \hat{\alpha}_{-n}$ and $(\hat{L}_n^\perp)^\dagger = \hat{L}_{-n}^\perp$, each term of the sum is also Hermitian, since
\begin{eqnarray*}
\left(i\left(\hat{\alpha}_{-n}^i\hat{L}_n^\perp-\hat{L}_{-n}^\perp\hat{\alpha}_{n}^i\right)\right)^\dagger&=& -i\left((\hat{L}_n^\perp)^\dagger(\hat{\alpha}_{-n}^i)^\dagger - (\hat{\alpha}_n^i)^\dagger(\hat{L}_{-n}^\perp)^\dagger\right)\\
&=& -i\left(\hat{L}_{-n}^\perp\hat{\alpha}_n^i - \hat{\alpha}_{-n}^i\hat{L}_n^\perp\right)\\
&=& i\left(\hat{\alpha}_{-n}^i\hat{L}_n^\perp - \hat{L}_{-n}^\perp\hat{\alpha}_n^i\right).
\end{eqnarray*}
The first term has a problem, as $\hat{x}_0^i$ and $\hat{p}^-$ do not commute. To fix this, we must symmetrize this term. 
\begin{equation}
\hat{\Lambda}^{i-} = \frac{\hat{x}_0^i \hat{p}^- + \hat{p}^-\hat{x}_0^-}{2}-\hat{x}_0^- \hat{p}^i +\frac{i}{\sqrt{2\alpha'}\hat{p}^+}\sum_{n=1}^\infty\frac{1}{n}\left(\hat{\alpha}_{-n}^i\hat{L}_n^\perp-\hat{L}_{-n}^\perp\hat{\alpha}_{n}^i\right).
\end{equation}
This symmetrization should seem reasonable, as our ansatz was based on the classical case where the two parameters did commute. We can now substitute out the $\hat{p}^-$ terms using \ref{lambdapminus}, giving
\begin{equation}
\hat{M}^{i-} = \frac{\hat{x}_0^i(\hat{L}_0^\perp +\lambda) + (\hat{L}_0^\perp+\lambda)\hat{x}_0^-}{4\alpha'\hat{p}^+}-\hat{x}_0^- \hat{p}^i +\frac{i}{\sqrt{2\alpha'}\hat{p}^+}\sum_{n=1}^\infty\frac{1}{n}\left(\hat{\alpha}_{-n}^i\hat{L}_n^\perp-\hat{L}_{-n}^\perp\hat{\alpha}_{n}^i\right).
\end{equation}  
We have reverted to calling this $\hat{M}^{i-}$ as this is the expression used to compute the commutator $[\hat{M}^{i-},\,\hat{M}^{j-}]$; our requirement that this commutator vanishes will now be determined using only the (real) parameter $\lambda$ and the dimension of spacetime $d$. The derivation of the commutator from here is not overly difficult, but it is rather long\cite{bering}. We only require the final result
\begin{equation}
[\hat{M}^{i-},\,\hat{M}^{j-}] = -\frac{1}{\alpha'(\hat{p}^+)^2}\sum_{n=1}^\infty\left(n\left(1-\frac{d-2}{24}\right)+\frac{1}{n}\left(\frac{d-2}{24}+\lambda\right)\right)\left(\hat{\alpha}_{-n}^i\hat{\alpha}_n^j-\hat{\alpha}_{-n}^j\hat{\alpha}_n^i\right).
\end{equation}
In general, the operator terms will not cancel, so the vanishing requirement is equivalent to saying that the constant terms,
\begin{equation}
n\left(1-\frac{d-2}{24}\right)+\frac{1}{n}\left(\frac{d-2}{24}+\lambda\right),
\end{equation}
must vanish for each $n$. Since there are only two unknowns, any two values of $n$ would be enough to uniquely determine $d$ and $\lambda$. Substituting $n=1$
\begin{equation}
1-\frac{d-2}{24}+\frac{d-2}{24}+\lambda = 1+\lambda = 0
\end{equation}
so we have fixed
\begin{equation}
\lambda = -1.
\end{equation}
Substituting this back in, and taking $n=2$
\begin{equation}
2\left(1-\frac{d-2}{24}\right)+\frac{1}{2}\left(\frac{d-2}{24}-1\right)=\frac{3}{2} -\frac{3}{2}\left(\frac{d-2}{24}\right) = 0,
\end{equation}
thereby fixing
\begin{equation}
d=26.
\end{equation}
We have found the dimensionality of spacetime in bosonic string theory! 

\section{Loose Ends and Oddities}\label{leo}
\subsection{A Brief Excursion into Number Theory}
Recall the origin of our parameter $\lambda$. It was defined in order to deal with a divergent sum. With its value now fixed, we are led to believe that
\begin{equation}
-1 = \frac{d-2}{2}\sum_{a=1}^\infty a,
\end{equation}
or, by a simple rearrangement, that the sum of the natural numbers is equal to $-1/12$. This mysterious claim is actually not completely unjustified. In some summation methods, for example the zeta-function regularization, this is indeed to value assigned to the sum of the natural numbers. To see how, consider the zeta function, defined as
\begin{equation}
\zeta(s) = \sum_{n=1}^\infty \frac{1}{n^s},\quad \text{Re}(s)>1.
\end{equation}
If we were to set $s=-1$, this would give the sum we are looking for. While, this value of $s$ does not satisfy the condition to the right of the definition, by it can be shown that by analytic continuation the sum is finite for all values of $s$ apart from $s=1$.

The gamma function is defined by
\begin{equation}
\Gamma(s) = \int_0^\infty u^{s-1}e^{-u}du,
\end{equation}
and substituting $nt=u$ gives
\begin{eqnarray}
\Gamma(s) &=& \int_0^\infty n^{s-1}t^{s-1}e^{-nt}ndt\nonumber\\
&=&\int_0^\infty n^{s}t^{s-1}e^{-nt}dt.
\end{eqnarray}
Multiplying this by the zeta-function gives
\begin{eqnarray}
\zeta(s)\Gamma(s) &=& \int_0^\infty t^{s-1}\sum_{n=1}^\infty e^{-nt}dt\nonumber\\
&=& \int_0^\infty t^{s-1}\sum_{n=1}^\infty (e^{-t})^ndt\nonumber\\
&=& \int_0^\infty \frac{t^{s-1}}{e^t-1}dt, \quad \text{Re}(s)>1\label{zetagamma}.
\end{eqnarray}
For small $t$ we can use the Taylor expansion for the denominator, namely
\begin{equation}\label{taylor}
\frac{1}{e^t-1} = \frac 1 t -\frac 1 2 + \frac t {12} + O(t^2).
\end{equation}
We can only use this expansion for $|t|<1$, so we split the integral \ref{zetagamma} in two, one integral for $t<1$, and one for $t>1$.
\begin{equation}
\zeta(s)\Gamma(s) = \int_0^1\frac{t^{s-1}}{e^t-1}dt + \int_1^\infty\frac{t^{s-1}}{e^t-1}dt
\end{equation}
The second integral converges for all $s\in\mathbb{C}$. Using the Taylor expansion we can write
\begin{equation}
\zeta(s)\Gamma(s) = \int_0^1\left(\frac{t^{s-1}}{e^t-1} -t^{s-1}\left(\frac1 t -\frac 1 2 +\frac t {12}\right) + t^{s-1}\left(\frac1 t -\frac 1 2 +\frac t {12}\right) \right)dt + \int_1^\infty\frac{t^{s-1}}{e^t-1}dt
\end{equation}
and integrate the third term under the first integral to find
\begin{equation}
\zeta(s)\Gamma(s) = \int_0^1\left(t^{s-1}\left(\frac{1}{e^t-1}- \frac1 t +\frac 1 2 -\frac t {12}\right)\right)dt + \frac1 {s-1} -\frac 1 {2s} +\frac 1 {12(s+1)} + \int_1^\infty\frac{t^{s-1}}{e^t-1}dt.
\end{equation}
Using \ref{taylor} the integrand of the first term is then $O(t^{s+1})$ so 
\begin{equation}
\int_0^1\left(t^{s-1}\left(\frac{1}{e^t-1}- \frac1 t +\frac 1 2 -\frac 1 {12}\right)\right)dt < \int_0^1 t^{s+1}dt,
\end{equation}
and for $\text{Re}(s)>-2$ the integral converges. That means that the right hand side is a valid analytic continuation of the function on the left hand side for $\text{Re}(s)>-2$. Then since $\Gamma(s)$ has removable poles at 0, -1, -2$\,\dots$, the function $1/\Gamma(s)$ has removable zeros at these values of $s$. These cancel out the removable poles of the right hand side at $s=0$ and $s=-1$. Thus, $\zeta(0) = 1/2$ and $\zeta(-1)=-1/12$, as required.
\subsection{The String Mass and Some Basic String States}
While constructing the state space of the string is outside the scope of this project, there are a few nice results we can obtain without too much work. 

Now that we have calculated the value of $\lambda$, we have the full form of the mass-squared operator, $\hat{M}^2$.
\begin{equation}
\hat{M}^2 = \frac{1}{\alpha'}\left(\sum_{i=2}^{25}\sum_{n=1}^\infty\hat{\alpha}_{-n}^i\hat{\alpha}_n^i-1\right)
\end{equation}
If we return to the $\hat{a}_n^i$ operators of section \ref{annihilation}, this becomes
\begin{equation}
\hat{M}^2 = \frac{1}{\alpha'}\left(\sum_{i=2}^{25}\sum_{n=1}^\infty n(\hat{a}_{n}^i)^\dagger\hat{a}_n^i-1\right),
\end{equation}
and it is clear that each of the combinations of operators in the sum is a number operator, similar to the one from the harmonic oscillator. It is helpful to give the sum of these terms, the total number operator a name. We can write
\begin{equation}
\hat{N}^\perp = \sum_{i=2}^{25}\sum_{n=1}^\infty n(\hat{a}_n^i)^\dagger\hat{a}_n^i.
\end{equation}
 One immediate consequence is that the ground states, states that are not excited by any creation operators $(\hat{a}_n^i)^\dagger$, will by annihilated by $\hat{N}^\perp$. These states will thus have mass-squared $=-1/\alpha'$. States which have negative mass-squared are called tachyons, and they are generally a sign of something going wrong. Tachyons being present in a theory means that the theory is inherently unstable. This is a fascinating area, with research still ongoing, as it has recently been shown that tachyons can appear even in the most sophisticated string theories.

The next point of interest comes from states with one creation operator with $n=1$. If we suppose a ground state $\ket{0}$, and consider the state $(\hat{a_1^j})^\dagger\ket{0}$ for some $2\leq j\leq25$, then applying $\hat{M}^2$ gives
\begin{eqnarray}
\hat{M}^2(\hat{a_1^j})^\dagger\ket{0} &=& \frac{1}{\alpha'}\left(\sum_{i=2}^{26}\sum_{n=1}^\infty n(\hat{a}_{n}^i)^\dagger\hat{a}_n^i-1\right)(\hat{a}_1^j)^\dagger\ket{0}\\
&=&\frac{1}{\alpha'}\left(\left(\hat{a}_1^j\right)^\dagger\hat{a}_1^j\left(\hat{a}_1^j\right)^\dagger\ket{0}-\left(\hat{a}_1^j\right)^\dagger\ket{0}\right)\\
&=&\frac{1}{\alpha'}\left(\left(\hat{a}_1^j\right)^\dagger\left(\left(\hat{a}_1^j\right)^\dagger\hat{a}_1^j+\left[\hat{a}_1^j,\,\left(\hat{a}_1^j\right)^\dagger\right]\right)-\left(\hat{a}_1^j\right)^\dagger \right)\ket{0}\\
&=&\frac{1}{\alpha'}\left(\left(\hat{a}_1^j\right)^\dagger-\left(\hat{a}_1^j\right)^\dagger\right)\ket{0}\\
&=& 0.
\end{eqnarray}
So the first set of excited states correspond to massless particles. These states turn out to be a remarkably good description of photons.

By continuing to add more creation operators we can generate general quantum states. To formally classify the space of these states is a little technical, but not too strenuous. A nice description can be find in Zwiebach\cite{zwiebach}.
\subsection{Hidden Dimensions}
A naive approach to the claim that a theory requires the dimension of spacetime to be 26 would be to see this as a contradiction. We can \emph{see} that spacetime is 4-dimensional, and it isn't entirely obvious where the other 22 would come from, or why we can't observe them.

One potential resolution is for the additional dimensions to be periodic. In a normal dimension as we understand it, all points are unique. Take the real line for example. Each point on the real line has its own unique value. But suppose this were not the case. For a periodic dimension we instead chose to equate all points that are separated by a distance of $2\pi$. So the point that was previously called $2\pi$ would be the same as the point previously called 0. In effect we have wrapped the real line around the edge of a circle of radius 1. By moving an integer multiple of $2\pi$ 'up' the real line we return to the same place.

Now instead of the real line, imagine we were on the outside of a cylinder of infinite length and radius 1. If we move \emph{along} the cylinder we can continue forever and never return to the same place. But if we instead move \emph{around} the cylinder then after moving a distance of $2\pi$ we will have returned to the spot where we started.

Attempting to continue this series of analogies is futile, as if we tried to add any additional dimensions we would have to try to picture structures in 4-dimensions, and humans aren't well equipped for that. But the principle remains the same. We could have some number (say four) conventional dimensions, and then some number (say 22) periodic, or \emph{compact} dimensions. 

It's not entirely clear yet how these dimensions could exist without us having noticed them yet. The key is in the size of the extra dimensions. Take a much reduced example. Suppose our observed spacetime was one dimensional, and there was one compact dimension of length $2\pi r$ along with it. Call these two dimensions $x$ and $y$ respectively. Now impose an infinite potential outside of an interval $x \in [0,a]$, creating a form of the classic square well problem\cite{gasiorowicz}.

The Schr\"{o}dinger equation for this situation is given by
\begin{equation}\label{sch4}
-\frac{\hbar^2}{2m}\left(\frac{\partial^2 \psi}{\partial x^2} + \frac{\partial^2  \psi}{\partial y^2}\right) = E\psi
\end{equation}
Using separation of variables suppose the solution is of the form $\psi(x,y)=\psi_1(x)\psi_2(y)$ for some $\psi_1$, $\psi_2$. Then by substituting in we see that we have two independent equations to solve,
\begin{eqnarray}
-\frac{\hbar^2}{2m}\frac{d\psi_1^2(x)}{dx^2} &=& E_1\psi^1(x)\\
-\frac{\hbar^2}{2m}\frac{d\psi_2^2(y)}{dy^2} &=& E_2\psi^2(y)
\end{eqnarray} 
where $E = E_1+E_2$. Notice that the equation for $\psi_1$ is exactly the same as if the $y$ dimension did not exist. The solutions of these differential equations with the given boundary conditions are
\begin{eqnarray}
\psi_1^p(x) &=& a_p\sin\frac{p\pi x}{a}\\
\psi_2^q(y) &=& b_q\sin\frac{qy}{r} + c_n\cos\frac{qy}{r}
\end{eqnarray}
where the superscripts on the left hand side index the solutions for integers $p$, $q$. In particular while (as normal) $\psi_1$ does not have a non-vanishing constant solution, by setting $q=0$ we obtain 
\begin{equation}
\psi_2^0(y) = c_0
\end{equation}
and so the overall solution to \ref{sch4} for $q=0$ and arbitrary integer $p$ is
\begin{equation}
\psi^{p0}(x,y) = c_0a_p\sin\frac{p\pi x}{a}.
\end{equation} 
which has energy 
\begin{equation}
E_{p,0} = \frac{\hbar^2p^2\pi^2}{2ma^2}.
\end{equation}
Notice that the presence of the $y$ dimension has not contributed at all. Now suppose that instead of $q=0$, the y-component of the wave function was in the first excited state, $q=1$.
The wave function would then be given by
\begin{equation}
\psi(x,y) = a_p\sin\frac{p\pi x}{a}\left(b_1\sin\frac{y}{r} + c_1\sin\frac{y}{r}\right),
\end{equation}
which has energy
\begin{equation}
E_{p,1} = \frac{\hbar^2}{2m}\left(\frac{p^2\pi^2}{a^2} + \frac{1}{r^2}\right).
\end{equation}

Now suppose that the length $r$ of the $y$ dimension is extremely small, in particular, much smaller than $a$, the width of the potential well. In this case since the terms in the energy are proportional to the inverse square of $a$ and $r$ respectively, the second term will contribute much more to the total energy. For example, suppose the well is 1 metre wide, and the y-dimension is 1 millimetre long ($10^{-3}$m). Then the $1/r^2$ term in the energy will be the larger of the two terms until $p$ reaches 319. Basically, it takes phenomenally large amounts of energy just to raise the y-component to the first excited state. In addition, as the y-dimension gets smaller the required energy to observe it grows even larger.

This is a very simplified example, with only two dimensions, and using the highly idealised square well situation. However, the same principles apply; for compact dimensions to become observable we might have to be using energies far in excess of our most up to date accelerators.
\section{Conclusion}
In this project we have dipped our toes into the ocean that is string theory. The most natural step for continued study would be to formally construct the state space, and consider the differences that are caused by considering closed instead of open strings.

Further out the possibilities are endless. Since the 1980s and the advent of supersymmetry new forms of string theory have been developed called superstring theories. In the context of this project it is interesting to note that the dimensionality of spacetime in superstring theories turns out to be 10.

Alternatively, there are many interesting results still available within bosonic string theory. For example there are many other ways of deriving the $d=26$ result. Quoting Bering\cite{bering}, apart from the derivation we used here there are three main other approaches:
\begin{itemize}
\item{No negative norm states/ghosts in the covariant formulation\cite{b5}}
\item{The vanishing of the conformal/Weyl anomaly in Polyakov's path integral formulation\cite{b6}}
\item{Nilpotency of the BRST generator in the covariant formulation\cite{freeman}}.
\end{itemize}
These other approaches have the advantage of not using the unintuitive zeta-regularization summation.
\section{Acknowledgements}
I would like to thank my project supervisor Ed Corrigan for useful discussion and feedback throughout the course of my project. I would also like to thank Peter Neuhaus for invaluable proof reading and assistance when the course of my project strayed too far into number theory for comfort.
% The following produces the References section. It should appear at the very end
% of your document.

\begin{thebibliography}{99}

\bibitem{zwiebach}, B. Zwiebach, {\em A First Course in String Theory}, 2nd ed, 2009, Cambridge University Press
\bibitem{gasiorowicz}, S. Gasiorowicz, {\em Quantum Physics}, 3rd ed, 2003, John Wiley \& Sons
\bibitem{peskin}, M.E. Peskin, D.V. Schroeder, {\em An Introduction to Quantum Field Theory}, 2005, Sarat Book House
\bibitem{strauss}, W.A. Strauss, {\em Partial Differential Equations, an Introduction}, 2nd ed, 2008, John Wiley \& Sons
\bibitem{frenkel}, I. Frenkel, J, Lepowsky, A. Meurman, {\em Vertex Operator Algebras and the Monster}, Chapter 1.9, 1989, Academic Press
\bibitem{analytical} L.N. Hand, J.D. Finch {\em Analytical Mechanics}, 1998, Cambridge University Press
\bibitem{ggrt} P. Goddard, J. Goldstone, C. Rebbi, C.B. Thorn, {\em Quantum Dynamics of the Massless Relativistic String}, 1972, Journal of Nuclear Physics B56 109-135
\bibitem{bering} K. Bering, {\em A Note on Angular Momentum Commutators in Light-Cone Formulation of Open Bosonic String Theory}, arXiv:1104.4446 [hep-th]
\bibitem{polchinski} J. Polchinski, {\em String Theory}, 1998, Cambridge University Press
\bibitem{gsw} M. Green, J. Schwarz, E. Witten, {\em Superstring Theory}
\bibitem{tong} D. Tong, {\em String Theory}, Cambridge University Lecture Notes, 2009
\bibitem{thooft} G. 't Hooft, {\em Introduction to String Theory}, Utrecht University Lecture Notes, 2009
\bibitem{b5} P. Goddard, C.B. Thorn, {\em Compatability of the Dual Pomeron with Unitarity and the Absence of Ghosts in the Dual Resonance Model}, 1972 Physics Letters B40 235-238
\bibitem{b6} A. M. Polyakov, {\em Quantum Geometry of Bosonic Strings}, 1981, Physics Letters B103 207-210
\bibitem{freeman} M.D. Freeman, D.I. Olive, {\em BRS Cohomology in String Theory and the No-Ghost Theorem}, 1986, Physics Letters B175 151-154
\end{thebibliography}

\end{document}
